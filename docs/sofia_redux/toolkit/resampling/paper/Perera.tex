\documentclass[preprint]{aastex63}
\usepackage{subfigure}
\usepackage{amsmath}
\usepackage{bm}
\usepackage{float}
\usepackage{natbib}
\setcounter{tocdepth}{4}
\setcounter{secnumdepth}{4}

\shorttitle{Resampling of Multi-Dimensional Data}
\shortauthors{Perera}

\begin{document}


\title{An Interpolation Method for Irregularly-Sampled Multi-Dimensional Data
       Cubes}


\author{Daniel Perera}
\affil{SOFIA-USRA, NASA Ames Research Center, MS 232--12, Building N232,
       PO Box 1, Moffett Field, CA 94035--0001, USA }
\email{dperera@sofia.usra.edu}


\begin{abstract}

We present a method of interpolating irregularly-sampled multi-dimensional data
onto arbitrary coordinates using local polynomial regression.
Local regression techniques fit models to a subset of samples in the vicinity
of a given coordinate, and may optionally weight samples according to
measurement error and/or relative position, resulting in a smooth,
self-consistent function that may be evaluated over the entire sample space.
Additionally, we derive an optional method for determining local kernel
estimators in a single step, approximating an optimal reduced chi-squared
statistic on the fit.

\end{abstract}


%% Keywords should appear after the \end{abstract} command. The uncommented
%% example has been keyed in ApJ style. See the instructions to authors
%% for the journal to which you are submitting your paper to determine
%% what keyword punctuation is appropriate.

\keywords{resampling, interpolation, local polynomial regression,
          multivariate polynomials, kernel estimation}

\section{Introduction}\label{sec:introduction}
Irregularly sampled data are commonplace in astronomy and other scientific
fields, where interpolation onto a regular spaced grid is almost always required
for visualization and analysis.
However, resampling data in three or more dimensions simultaneously is
complicated and computationally expensive, so frequently one resorts to
resampling multiple times in a series of steps for each dimension.
This process can be time-consuming and prone to propagating errors, introduced
in the interpolation procedure, from one dimension into others.
It also does not account for the continuity requirements on adjacent values
in the multiple dimensions imposed by the physical data collection process.
In this paper, we outline a method to carry out simultaneous resampling
(interpolation) in multiple dimensions.

\subsection{Motivation}\label{subsec:motivation}

The FIFI-LS instrument~\citep{Colditz18, Fischer18} used on SOFIA is an example
of an integral field spectrograph that generates data cubes in which the
original values are not evenly sampled on a regular three-dimensional grid.
As described by~\citet{Fischer18}, FIFI-LS records flux values at an array of
discrete 25 $x$ and $y$ positions on the sky (spaxels), and at an array of
16 wavelengths $\lambda$.
The spaxels are not regularly spaced on the sky.
Furthermore, the set of sampled wavelength values varies from spaxel to spaxel,
and some measurements are either missing or noisy.
~\citet{Vacca20}
\footnote{See also the \href{https://www.sofia.usra.edu/sites/default/files/USpot_DCS_DPS/Documents/FIFI-LS_GO_Handbook_RevC.pdf}
{FIFI-LS Observer's Handbook} and
\href{https://www.sofia.usra.edu/sites/default/files/USpot_DCS_DPS/Documents/FIFI-LSDataHandbookRevE.pdf}
{The FIFI-LS Guest Observer (GO) Data Handbook}.}
give a brief overview of the pipeline developed to convert instrumental values
recorded by the instrument into flux estimates at regularly sampled spatial and
wavelength coordinates.
The initial version of this data processing pipeline dealt with the irregularly
sampled cloud of data points in three dimensions by first interpolating the
flux values in wavelength at each spatial point onto a regular array, so that
each spaxel was assigned the identical wavelength array.
Then at each wavelength, interpolation was performed again in the spatial
dimension to compute fluxes on a regular spatial grid.
In this way, a regularly sampled data cube in three dimensions (space and
wavelength) was generated.
Although this procedure is effective, it can be slow, and the wavelength
interpolation ignores the coupling (due to physical continuity requirements) of
data points that are spatially adjacent (and therefore within the instruments
point spread function).

\subsection{Local Polynomial Regression (LPR)}\label{subsec:LPR}

One of the most employed and well understood methods of interpolating data is
to model sample measurements as a smooth function of unknown parameters, and
then estimate a set of parameters that best fit those data using regression.
For the purposes of this paper, polynomials functions are used for their
versatility and ease of use.

One could choose to model the entire sample space with a single set of
parameters that describe the ``True'' underlying function of the noisy sample
measurements.
However, as the complexity of the structure to be modelled increases, so too
does the number of parameters required, leading to numerous difficulties.
Numerical instability becomes more pronounced in higher dimensions and for
higher polynomial orders (\citet{Noferini2016}).
~\citet{Gelman19} show that for polynomials, high order fits have increased
sensitivity to noise, and strong dependence on the order of polynomial fit.
It is therefore preferable to derive numerous model from subsets of samples in
the local vicinity of each point at which a fit is required.

LPR is discussed in detail by~\citep{Fan96a}, with multivariate extensions
provided by~\citep{Masry1996}, and expanded upon by~\citep{Gu15}.
All methods essentially minimize the following problem via weighted least
squares given by~\citep{Fan96b} as

\begin{equation}
       \underset{\hat{c}}{min} \sum_{i=1}^{N}{
       \left( y_i - \sum_{j=0}^{p}{c_j (x_i - v)^j}\right)^2}
       Wt(\frac{x_i - v}{h})
       \label{eq:equation70}
\end{equation}

where $(x_1, y_1), \cdots,(x_N, y_N)$ form an i.i.d sample from a certain
population, $Wt$ is a non-negative, symmetric, and bounded weighting function
with smoothing parameter (or bandwidth) $h$, and $v$ is the point of interest.
$h$ may be optimized globally or locally for each resampling point by balancing
the bias and variance of $\hat{c}$ such that the optimal bandwidth minimizes
the Mean Squared Error (MSE) of $\hat{c}(x) - c(x)$

\begin{equation}
       \hat{h} = arg\, \underset{h}{min}\, \hat{\text{MSE}}(f(\hat{c}, \bm{x}))
       \label{eq:equation69}
\end{equation}

Unfortunately, this is not always effective when resampling irregular data
taken from real observations.
The optimal bandwidth given by~\citep{Gu15} assumes that $x$ are i.i.d, with
a common density, and errors on the sample measurements ($\epsilon$) are likewise
i.i.d with zero mean, and independent of $x$.
These assumptions cannot always hold for FIFI-LS observations on SOFIA when
data are combined over multiple flights over long periods of time, leading to
uneven sampling density, and calibration errors that can vary substantially
between samples.

Additionally, little attention has been given to defining what constitutes the
``local'' subset of samples in a fit.
While the standard method of selecting the N-nearest neighbors is acceptable
if $x$ is i.d.d, doing so for irregularly sampled data in which there is no such
guarantee is a dangerous tactic to employ.
For FIFI-LS observations, the spatial and spectral response of the telescope is
known, it is scientifically irresponsible to perform regression on samples for
which there is no coupling to the response function.
In higher dimensions, the problem of determining local sample subsets using
nearest-neighbors becomes increasingly severe as there is no accounting for
sufficient sampling along a given dimension, potentially leading to unstable
solutions and co-linearity.
The following LPR implementation attempts to overcome many of these
difficulties while remaining computationally viable.


\section{Formulation}\label{sec:formulation}
\subsection{Polynomial Representation of K-Dimensional Data}
\label{subsec:polynomial-representation-of-k-dimensional-data}

We assume that the input data consist of $N$ measurements (samples) $y$
distributed over $K$ dimensions and that the `true' (but unknown) generating
function for the measured values is given by $f$ which can be adequately
represented by a K-dimensional polynomial function whose order $o_k$ is allowed
to vary in each dimension $k$.
Then at the coordinate $\bm{x}$ we have,

\begin{equation}
    f(\bm{x}) = \sum_{p_1 = 0}^{o_1}
                \sum_{p_2 = 0}^{o_2}
                \cdots \sum_{p_K = 0}^{o_K}
                \lambda_{(p_1, p_2,\cdots p_K)}
                c_{(p_1, p_2,\cdots p_K)}
                x_1^{p_1} x_1^{p_1} \cdots x_K^{p_K}\label{eq:equation}
\end{equation}
Here, $c$  are the coefficients of the polynomial and $p$ are the exponents of
the coordinate variables $x$ for the $K$ dimensions, while
$\lambda_{(p_1, p_2,\cdots p_K)}$ is defined as

\begin{equation}
    \lambda_{(p_1, p_2,\cdots p_K)} = \left\{
        \begin{array}{rl}
            1, & \textrm{if } \sum_{k = 0}^{K} ~ p_k \leq max(o) \\
            0, & \textrm{otherwise.}
        \end{array} \right.
    \label{eq:equation2}
\end{equation}

We can then define a set $S$ containing all polynomial exponent combinations
where $\lambda = 1$,

\begin{equation}
    S = \{ (p_1, p_2,\cdots p_K) | \lambda_{(p_1, p_2,\cdots p_K)} = 1 \}
    \label{eq:equation3}
\end{equation}

and can then re-write equation~\ref{eq:equation} in the form

\begin{equation}
    f(\bm{x})  =  \sum_{s \in S} c_s \Phi_s\label{eq:equation4}
\end{equation}

where the polynomial terms of the equation are given as

\begin{equation}
    \Phi_s = \prod_{k = 1}^K x_k^{s_k}
    \label{eq:equation5}
\end{equation}

The set $s$ is in lexographic order such that (for $K=3$) $\{0,1,1\}$ comes
before $\{1, 0, 0\}$.
As an example, when $K=2$ and $o_k = 2$ for all $k$ we have

\begin{align}
    s_{(p=2,K=2)} &= [(0,0), (0,1), (0,2), (1,0), (1,1), (2,0)] \nonumber \\
    \Phi          &= [1, x_2, x_2^2, x_1, x_1 x_2, x_1^2] \nonumber \\
    f(\bm{x})     &= c_1 + c_2 x_2 + c_3 x_2^2 + c_4 x_1 +
                     c_5 x_1 x_2 + c_6 x_1^2. \nonumber
\end{align}

Similarly, for $K=3$ where $p_1 = 1$, $p_2 = 2$, and $p_3 = 3$ we have
\begin{align}
    s_{(p=[1,2,3],K=3)} &=&
        &[(0,0,0), (0,0,1), (0,0,2), (0,0,3), (0,1,0), (0,1,1), (0,1,2), \nonumber \\
        && &(0,2,0), (0,2,1), (1,0,1), (1,0,2), (1,1,0), (1,1,1), (1,2,0)] \nonumber \\
    \Phi &=&
        &[1, x_3, x_3^2, x_3^3, x_2, x_2 x_3, x_2 x_3^2, x_2^2, x_2^2 x_3,
         x_1 x_3, x_1 x_3^2, x_1 x_2, x_1 x_2 x_3, x_1 x_2^2] \nonumber \\
    f(\bm{x}) &=&
        &c_1 + c_2 x_3 + c_3 x_3^2 + c_4 x_3^3 + c_5 x_2 +
        c_6 x_2 x_3 + c_7 x_2 x_3^2 + c_8 x_2 + c_9 x_2^2 x_3 + \nonumber \\
        &&&c_{10} x_1 x_3 + c_{11} x_1 x_3^2 + c_{12} x_1 x_2 +
        c_{13} x_1 x_2 x_3 + c_{14} x_1 x_2^2 \nonumber
\end{align}

\subsection{Resampling Algorithm}\label{subsec:resampling-algorithm}

With the above representation for $f$ we now wish to evaluate $f(\bm{x})$ at a
point $\bm{x} = \bm{v}$.
We assume that $f$ has been sampled at $N$ data (or measurement) points
$\bm{y}_i$, $i = 1,\cdots,N$.
Throughout, we refer to the measurement values $\bm{y}_i$ at coordinates
$\bm{x}_i$ as samples, and the coordinates $\bm{v}$ as resampling points.
To compute $f(\bm{v})$ we consider the $N$ data values within a window region
set $\Omega$ centered on $\bm{v}$.
This set comprises values within an ellipsoidal hyper-surface.
We assume each measured value $\bm{y}_i$ has an associated noise $\epsilon_i$.
In this case,

\begin{equation}
    f(\bm{x}_i) + \epsilon_i = y_i,\label{eq:equation6}
\end{equation}

and the task of determining $f(\bm{v})$ becomes one of estimating the $S$
coefficients $c$ of $f$ from $N$ noisy measurements.
Using the above notation we have

\begin{equation}
    \mathbf{c} \cdot \mathbf{\Phi} + \bm{\epsilon} = \mathbf{y}
    \label{eq:equation7}
\end{equation}

which can be expressed in matrix notation as

\begin{equation}
    \begin{bmatrix}
        1 & \Phi_{(2,1)} & \Phi_{(3,1)} & \cdots & \Phi_{(S,1)} \\
        1 & \Phi_{(2,2)} & \Phi_{(3,2)} & \cdots & \Phi_{(S,2)} \\
        1 & \Phi_{(2,3)} & \Phi_{(3,3)} & \cdots & \Phi_{(S,3)} \\
        \vdots & \vdots & \vdots & \ddots & \vdots \\
        1 & \Phi_{(2,N)} & \Phi_{(3,N)} & \cdots & \Phi_{(S,N)}
    \end{bmatrix}
    \quad
    \begin{bmatrix}
        c_1 \\
        c_2 \\
        \vdots \\
        c_S
    \end{bmatrix}
    +
    \begin{bmatrix}
        \epsilon_1 \\
        \epsilon_2 \\
        \epsilon_3 \\
        \vdots \\
        \epsilon_N
    \end{bmatrix}
    =
    \begin{bmatrix}
        y_1 \\
        y_2 \\
        y_3 \\
        \vdots \\
        y_N
    \end{bmatrix}
    \label{eq:equation8}
\end{equation}

where $\Phi_{(m,i)}$ is the $m^{th}$ element of the set of $\Phi$ for sample point
$i$.
We now wish to solve for the set of $S$ coefficients $c$.
The standard procedure is to use the least-squares formalism, by which the
estimated values for the coefficients, $\hat{c}$, are given by~\citet{Fan96b}
as

\begin{equation}
    \mathbf{\hat{c}} = (\mathbf{\Phi}^T\mathbf{\Phi})^{-1}
                       \mathbf{\Phi}^T \mathbf{y}
    \label{eq:equation9}
\end{equation}

or, in the case of a weighted fit,

\begin{equation}
    \mathbf{\hat{c}} = (\mathbf{\Phi}^T \mathbf{W} \mathbf{\Phi})^{-1}
                       \mathbf{\Phi}^T \mathbf{W} \mathbf{y}
    \label{eq:equation10}
\end{equation}

where $\mathbf{W}$ is an $N \textrm{x} N$ diagonal weight matrix in which the
diagonal elements are the weights for each measurement point.
Once the $\hat{c}$ values have been determined,  $f(v)$ can be evaluated:

\begin{align}
    \Phi_s(\bm{v}) &= \prod_{k=1}^K v_k^{s_k} \\
    f(\bm{v}) &= \sum_{s \in S} \hat{c}_s \Phi_s(\bm{v})
\end{align}

The conditional bias and variance on $\hat{c}$ are given by~\citet{Fan96b} as

\begin{align}
    \text{Bias}(\hat{c} | \Phi) &= E(\hat{c} | \Phi) - c =
        (\Phi^T W \Phi)^{-1} \Phi^T W r \label{eq:equation71} \\
    \text{Var}(\hat{c} | \Phi) &= C_{\hat{c}} =
        (\Phi^T W \Phi)^{-1} (\Phi^T \Sigma \Phi) (\Phi^T W \Phi)^{-1}
        \label{eq:equation72}
\end{align}

where the residuals on the fit are given as

\begin{equation}
    r = \Phi \hat{c} - y
    \label{eq:equation73}
\end{equation}

and

\begin{equation}
    \Sigma = {diag}({w_\delta^2}(x_{i,m}) \sigma_x^2(x_i))
    \label{eq:equation74}
\end{equation}

$\sigma_x(x)$ is the conditional variance of $y$ given $x$, and
${w_\delta}(x_{i,m})$ is the positional (distance) weighting for sample
$i$ with respect to resampling point $m$ (equivalent to the $Wt$ function in
equation~\ref{eq:equation70}).

\subsection{Error Propagation and Estimation}\label{subsec:errors}

The Mean Squared Error (MSE) of a fit at point $v$ is given by~\citet{Fan96b}
as

\begin{equation}
    \text{MSE}\{f(v)\} = (\text{Bias}\{f(\bm{v})\, |\, \hat{c}_v \})^2 +
                          \text{Var}\{f(\bm{v})\, |\, \hat{c}_v \}
    \label{eq:equation20}
\end{equation}

and variance on the fit at $\Phi(v)$ may be propagated as

\begin{equation}
    \text{Var}\{f(\bm{v})\, |\, \hat{c}_v \} = \Phi(v)^T C_{\hat{c}} \Phi(v)
    \label{eq:equation24}
\end{equation}

We also derive a data driven estimate of the variance-covariance matrix from
the residuals on the fit $r$ using

\begin{equation}
    \text{Var}\{ \hat{c}\, |\, \Phi, r \} = R_{\hat{c}} =
        \chi_r^2 (\Phi^T W \Phi)^{-1}
    \label{eq:equation22}
\end{equation}

where $\chi_r^2$ is the reduced chi-squared statistic calculated as

\begin{equation}
    \chi_r^2 = \frac{|\Omega|}{tr(W)(|\Omega| - |S|)} r^T W r
    \label{eq:equation75}
\end{equation}

Note that $|S|$ is the total number of parameters that are being estimated in
for the fit, where $S$ is given in equation~\ref{eq:equation3}, and $|\Omega|$
are the total number of samples in the ``local'' region of $v$, so the quantity
$|\Omega| - |S|$ gives the degrees of freedom in our fit.
The variance on a fit at $v$ according to residuals is then given as

\begin{equation}
    \text{Var}\{f(\bm{v})\, |\, r\} = \Phi(v)^T R_{\hat{c}} \Phi(v)
    \label{eq:equation76}
\end{equation}


\section{Implementation}\label{sec:implementation}
The resampling algorithm takes $N$ samples with coordinates $\bm{x}$ and
generates local polynomial fits at $M$ resampling points with coordinates
$\bm{v}$.
In this case, the ``local'' region for a single resampling point $m$ is defined
by a K-dimensional ellipsoid centered on $\bm{v}_m$.
The principle axes of the ellipsoid are given as a user supplied parameter
$\omega$ so that equation of the ellipsoid centered on point $m$ is given as

\begin{equation}
    \sum_{k=1}^{K}{\frac{(x_k - v_{m, k})^2}{\omega_k^2}} = 1
    \label{eq:equation11}
\end{equation}

The window region of point $m$ contains the set of samples $\Omega_m$ where sample
$i$ is included when

\begin{equation}
    i \in \Omega_m ~ \vert ~
        \sum_{k=1}^{K}{\frac{(x_{i, k} - v_{m, k})^2}{\omega_k^2}} \leq 1
    \label{eq:equation12}
\end{equation}

This window region will be constant for all resampling points.
Therefore, it should be chosen carefully as it defines the subset of samples
that are to be considered ``local'' for the LPR of each resampling point.
Although there is no strict limit to the size of the window in each dimension,
it should be chosen so that there are sufficient samples to perform a
polynomial fit of the desired order in most cases (see
section~\ref{subsubsec:order-rejection}).
Since $\omega$ is constant, determining $\omega$ for measurements consisting of
variable sample densities can be difficult.
It may be necessary to set $\omega$ to a large value in order to ensure fits occur
at certain resampling points.

If the instrumental response of the observing device is of importance, then
the window size should be chosen to represent a maximum distance for any
coupling effects.
For example, FIFI-LS reductions use a default window of
$3 \times \text{fwhm}_{\bar{\lambda}}$ in the spatial dimensions.
If distance weighting is enabled (see section~\ref{subsec:weighting}), and
$\sigma_x \ll \omega$, there is no significant consequence for the final result
with increased $\omega$.
Computationally, however, the window size increases the number of operations by
$\mathcal{O}(\omega^{K})$ assuming uniform sampling density, and should be
given extra consideration in higher dimensional reductions.
Figure~\ref{fig:resampling_window} gives a graphical representation of the
window in 3 dimensions.

\subsection{Search Problem}\label{subsec:search-problem}

In many cases, both $M$ and $N$ are large, and deriving $\Omega$ for all $M$
resampling points is computationally expensive.
To shorten processing time, the reduction is performed in parallel with each
thread performing a search on set of samples that are guaranteed to be close to
another set of resampling points.
This is accomplished by dividing up the sample coordinate space into regularly
spaced K-orthotopes (blocks) with the width in dimension $k$ equal to
$\omega_k$.
Binning resampling points and samples into blocks is a quick and easy procedure.
For any given block, the search for samples within the window region of any
resampling point in that block should be limited to the block itself, and all
immediately neighboring blocks as shown in figure~\ref{fig:block_neighbors}.

\begin{figure}[H]
  \begin{center}
  \includegraphics[width=0.5\linewidth]{images/window.png}
  \caption{The window region centered around point $\bm{v}_m$ in 3-dimensions.
           All samples falling within the shaded region will be included in
           the fit.  The principle axes ($\omega$) of the ellipsoid defining
           the bounds of the window are supplied by the user.}
  \label{fig:resampling_window}
  \end{center}
\end{figure}

\begin{figure}[H]
  \begin{center}
  \includegraphics[width=0.5\linewidth]{images/block_neighbors.png}
  \caption{Division of a 2-D sample space into blocks.  The search for
           samples inside the window region of all resampling points within
           the target block is limited to immediately neighboring blocks.}
  \label{fig:block_neighbors}
  \end{center}
\end{figure}

The procedure used to bin samples and points into blocks simply by assigning an
integer index to any given coordinate $\bm{x}$ in K-dimensions using

\begin{equation}
    \text{block}_{i, k} = \left\lfloor
        \frac{x_{i,k} - min(\bm{x_k})}{\omega_k}
    \right\rfloor
    \label{eq:equation13}
\end{equation}

for coordinate $i$ in dimension $k$ where $\lfloor\cdots\rfloor$ indicates the
floored value.
All neighboring blocks (the ``neighborhood'') including the block itself may
then be found by simply applying the kernel

\begin{equation}
    \text{neighborhood}_{i, k} = \text{block}_{i, k} + [-1, 0, 1]
    \label{eq:equation14}
\end{equation}

Once the block indices for all samples and points have been found, reductions
are performed in parallel over all blocks.
Each thread in the parallel reduction is sent a block of resampling points
along with its neighborhood of samples as shown in
figure~\ref{fig:parallel_reduction}.
Every resampling point in a thread is then processed in serial, first
determining which samples in the neighborhood are inside the window region
$\Omega_m$ of point $m$, and then deriving a fitted value at that location
(figure~\ref{fig:single_thread_reduction}).

\begin{figure}[H]
  \begin{center}
  \includegraphics[width=0.8\linewidth]{images/parallel.png}
  \caption{Left: Samples (blue dots) and resampling points (red crosses)
           divided into blocks for parallel reduction.  Right: Four threads
           of the parallel reduction, each containing a single block of
           resampling points (red grid squares) and its surrounding
           neighborhood of samples (blue grid squares).}
  \label{fig:parallel_reduction}
  \end{center}
\end{figure}

\begin{figure}[H]
  \begin{center}
  \includegraphics[width=\linewidth]{images/single_thread_reduction.png}
  \caption{Left: A single thread of the parallel reduction containing a set
           of resampling points (crosses) and samples (blue dots).  Each
           resampling point will be processed in series.  Right: All samples
           within the window region ($\Omega_m$) of point $m$ at coordinate
           $v_m$ will be included in the fit.}
  \label{fig:single_thread_reduction}
  \end{center}
\end{figure}

\subsection{Distribution Checks}\label{subsec:distribution-checks}

Before deriving a fit from the samples in the window region $\Omega_m$ of
about coordinate $v_m$ of resampling point $m$, a number of checks may be
performed.
These ensure that a polynomial fit of the desired order is possible, and the
position of the resampling point relative to the sample distribution will
result in a representative fit.

\subsubsection{Edge Rejection}\label{subsubsec:edge-rejection}

Polynomial fits (especially higher order polynomials) generally become
increasingly less representative of the samples from which they were derived
away from the center of the sample distribution.
A simple one-dimensional example shown in figure~\ref{fig:distribution_error_1d}
illustrates this deviation.

Away from the center of the distribution ($\bar{x} = 0$), both the deviation
and error on the fit begin to increase significantly, and it would be
undesirable to attempt a fit in such regions.
Therefore, the user may supply an ``edge threshold'' parameter
($\beta_{edge} > 0$) effectively defining an ``edge'' around the center of the
distribution.
No fitting will be permitted for any resampling point located outside of this
edge, and the value of the fit for any rejected point will be set to a
user-defined failure value.

\begin{figure}[H]
  \begin{center}
  \includegraphics[width=\linewidth]{images/distribution_error_1d.png}
  \caption{A $3^{rd}$ order polynomial fit (red) to the samples (blue).  The
           shaded red area is the $1\sigma$ error on the fit.}
  \label{fig:distribution_error_1d}
  \end{center}
\end{figure}

For a single dimension, we define the edge threshold parameter such that

\begin{equation}
    f(x)=\begin{cases}
        \hat{c}_m \cdot \Phi(v_m), & \textrm{if }
             | v_m - \bar{x} | \leq \sigma_x / \beta_{edge} \\
         \textrm{Failure value}, & \textrm{otherwise.}
    \end{cases}
    \label{eq:equation15}
\end{equation}

As $\beta_{edge}$ increases, the acceptable fitting region becomes more
concentrated about the center of the distribution ($\bar{\bm{x}}$).
In multiple dimensions, the covariance of the sample distribution
($\bm{\Sigma_x}$) is used, allowing the additional benefit of preventing fits
away from colinear type distributions.
For $K > 1$, the following exclusion is applied:

\begin{equation}
    f(x)=\begin{cases}
        \hat{c}_m \cdot \Phi(v_m), & \textrm{if }
             \sqrt{(\bm{v}_{m} - \bar{\bm{x}})^{T}
             {\Sigma_x}^{-1}(\bm{v}_m - \bar{\bm{x}})}
             \leq \beta_{edge}^{-1}\\
         \textrm{Failure value}, & \textrm{otherwise.}
    \end{cases}
    \label{eq:equation16}
\end{equation}

The point rejection region for a 2-dimensional sample distribution is
illustrated in figure~\ref{fig:distribution_error_2d} using an edge threshold
parameter $\beta_{edge}$ = 0.5, rejecting any fits occurring at more than
$2\sigma_x$ from the center of the sample distribution.

\begin{figure}[H]
  \begin{center}
  \includegraphics[width=0.75\linewidth]{images/distribution_error_2d.png}
  \caption{Edge rejection with $\beta_{edge}=0.5$.  No fits will be attempted
           more than $2\sigma_x$ from the center of the sample distribution
           (blue dots).  The fit exclusion area is shaded red and lines of
           constant $\sigma_x$ (Mahalanobis distance) are plotted as red dashed
           lines.}
  \label{fig:distribution_error_2d}
  \end{center}
\end{figure}

\subsubsection{Polynomial Order Rejection}
\label{subsubsec:order-rejection}

For the terms of a polynomial equation to be linearly independent over each
dimension, we require that there be at least $n_{order}$ samples to fit for
a given order $\bm{o}$, where

\begin{equation}
    n_{order} = \prod_{k=1}^{K}{(o_k + 1)}
    \label{eq:equation17}
\end{equation}

For example, two samples can uniquely define a first order (linear)
1-dimensional polynomial so long as those samples occupy unique coordinates.
However, those same two points can be modelled exactly by an infinite number of
higher order polynomials.
There is currently one mandatory and two optional polynomial order checks.
In all cases it is mandatory that

\begin{equation}
    | \Omega_m | \geq n_{order}
    \label{eq:equation18}
\end{equation}

where $| \Omega_m |$ indicates the cardinality (size, or number of samples) of
the set $\Omega_m$ for point $m$.
If the above condition is not met, the fit is aborted, and a failure value is
returned for the value at $v_m$.
In general, we want a ``best fit'' solution (in a least-squares sense) where
$| \Omega_m | > n_{order}$.
However, equation~\ref{eq:equation18} only requires that an exact fit is
possible.
If the user can guarantee that the samples are uniformly distributed, then
the procedure may be stopped here to save processing time.

The second step checks the number of unique coordinates in each dimension $k$
such that

\begin{equation}
    | \{ x_{i, k} \forall i \in \Omega_m \} | > o_k
    \label{eq:equation19}
\end{equation}

where $|\{\cdots\}|$ gives the cardinality (number of unique values) of the
sample coordinates for a given dimension.
The above check ensures that we have enough unique terms to allow for matrix
inversion in equations~\ref{eq:equation9} and~\ref{eq:equation10}.
Depending on how the user has defined $\beta_{edge}$
(see~\ref{subsubsec:edge-rejection}), this may allow for extrapolation at any
point within the window.

The final check ensures that there are enough samples to provide for a
least-squares best fit solution, and that any fitting occurs within the bounds
defined by the samples.
For a fit to occur,

\begin{align}
    |\{x_{i, k} \forall i \in \Omega_m \, \vert x_{i, k} < v_{m, k}\}| \geq o_k
    \nonumber \\
    |\{x_{i, k} \forall i \in \Omega_m \, \vert x_{i, k} > v_{m, k}\}| \geq o_k
    \label{eq:equation:20}
\end{align}

ensuring that $|\Omega_m| \geq 2 n_{order}$.

Examples of these three order checking algorithms are shown in
figure~\ref{fig:order_check}.
If the sample distribution does not meet the criteria defined in the order
checking algorithm, the returned fit value will be set to a user defined
failure value.

\begin{figure}[H]
  \begin{center}
  \includegraphics[width=\linewidth]{images/order_check.png}
  \caption{The three available edge checking algorithms shown in order of
           increasing robustness from left to right.  Samples (black dots)
           are regularly spaced at $\Omega / 2$ in the horizontal direction,
           and $\Omega / 6$ in the vertical direction.  Regions in which
           fits may occur are shaded green.}
  \label{fig:order_check}
  \end{center}
\end{figure}


\subsection{Weighting}\label{subsec:weighting}

Weighted polynomial regression is a method by which some samples in the local
polynomial fit have a stronger influence on the solution of the best fit
coefficients, as defined in equation~\ref{eq:equation10}.
Weighting may be based on the measurement error associated with each sample,
the position of a sample relative to each resampling point, or a combination of
both error and position.

The weighting applied to point $m$ is given as

\begin{equation}
    diag(W_m) = w_{\sigma, i} w_{\delta}(x_{i, m})\, \vert \, i \in \Omega_m
    \label{eq:equation21}
\end{equation}

where $\Omega_m$ is the local subset of samples included in the fit for point
$m$.
If positional (distance) weighting is not applied, then $w_\delta = 1$, and
error weighting is disabled when $w_\sigma=1$.
If no weighting occurs, then $W = I$, where $I$ is an identity matrix of rank
$|\Omega_m|$, and the notation $|.|$ indicates the cardinality (size or number
of unique values) in the set.

\subsubsection{Error Weighting}\label{subsubsec:error-weighting}

If the $1\sigma$ measurement error of sample $i$ is known ($\sigma_{y_i}$), the
associated weighting factor is:

\begin{equation}
    w_{\sigma, i} = \frac{1}{\sigma_{y_i}^2}
    \label{eq:equation23}
\end{equation}

The following figure (\ref{fig:error-weighting}) shows an example of fitting
with and without error weighting.

\begin{figure}[H]
  \begin{center}
  \includegraphics[width=0.75\linewidth]{images/error_weighting.png}
  \caption{An example of the application of error weighting in 1-dimension.
           Two of the samples shown with large error bars will skew the
           fit away from the underlying function if errors are not accounted
           for.}
  \label{fig:error-weighting}
  \end{center}
\end{figure}

\subsubsection{Distance Weighting}
\label{subsubsec:distance-weighting}

Distance weighting applies higher significance to samples that are closer to
the resampling point.
For the solution to converge effectively, the weighting kernel should be a
non-negative, bounded probability function (\citet{Gu15}).
Two popular functions are the Epanechnikov and Gaussian kernels, of which we
have opted for the Gaussian function as it has some desirable qualities that
are exploited during the adaptive weighting formulation (see
section~\ref{subsec:adaptive-weighting}).
The distance weighting applied to a single sample $i$ at coordinate $x_i$ for
a local polynomial fit centered on resampling point $m$ at coordinate $v_m$ is
given as

\begin{equation}
        w_{(\delta, i, m)} =
            exp \left(
                   -\sum_{k=1}^{K}{\frac{(v_{m, k} - x_{i, k})^2}{\alpha_k}}
                \right)
        \label{eq:equation25}
\end{equation}

where $\alpha$ is a user supplied smoothing parameter which may vary for each
dimension ($k$).
As $\alpha \rightarrow 0$, closer samples gain increasing influence over the
fit, whereas larger $\alpha$ values result in more equal weighting between
samples, irrespective of distance.
Note that in terms of the bandwidth given in equation~\ref{eq:equation70},
$h = \sqrt{\alpha}$.
We also frequently use the equation defining a multivariate Gaussian where

\begin{equation}
    w_{\delta} = exp \left(-(v - x)^T A^{-1} (v - x) \right)
    \label{eq:equation77}
\end{equation}

where $A = diag(2 \sigma_w^2) = diag(\alpha)$, and $\sigma_w$ is the standard
deviation of the Gaussian.
An example of distance weighting, and the effect of modifying $\alpha$ can be
seen in figure~\ref{fig:standard-distance-weighting}.

\begin{figure}[H]
  \begin{center}
  \includegraphics[width=0.75\linewidth]{images/standard_distance_weighting.png}
  \caption{The effect of distance weighting on a second order polynomial fit to
           samples generated using $y=\cos(\pi x)$.  The smoothing parameter
           ($\alpha$) can be modified to increase the influence of samples in
           the fit based on distance from the resampling point ($\Delta x$).}
  \label{fig:standard-distance-weighting}
  \end{center}
\end{figure}

\subsection{Adaptive Weighting}\label{subsec:adaptive-weighting}

If the data we wish to model can be easily represented by a smooth function,
then standard distance weighting may produce an acceptable fit.
However, when the underlying function contains a mix of structures of
varying complexity, a single distance weighting parameter may not be adequate.
This is alleviated somewhat by allowing the bandwidth parameter $h$ in
equation~\ref{eq:equation70} to vary at each resampling point, and is
extensively discussed by~\citet{Fan96b} and~\citet{Masry1996} amongst others.

Most methods seek to find the optimal smoothing parameter by minimizing
the mean squared error given in equation~\ref{eq:equation20}, balancing
the leading bias and variance.
In the case where we have supplied measurement errors, this does not always
result in what one may consider an ``optimal fit''.
This is especially true in the case of FIFI-LS where minimizing the overall
MSE results in highly spurious fits.
An alternative presented here, defines the optimal fit as one in which the
reduced chi-squared statistic ($\chi_r^2$), given in
equation~\ref{eq:equation75}, is equal to one.
When the measurement error is known, $\chi_r^2 > 1$ indicates a poor fit which
does not adequately represent the data.
Conversely, a fit where $\chi_r^2 < 1$ is over-fitting the data, and can be
thought of as fitting to the noise.
Therefore, our goal in determining an optimal smoothing parameter is to get
$\chi_r^2 \to 1$, rather than minimizing the MSE, which could likely result
in $\chi_r^2 \ll 1$.

We accomplish this by performing an initial test reduction using both distance
and error weighting where

\begin{equation}
    W = W_{\delta} W_{\sigma} = diag(w_{\delta, i} w_{\sigma, i}) =
        diag(\frac{w_{\delta, i}}{\sigma_{y, i}^2})
    \label{eq:equation27}
\end{equation}

and propagate the theoretical and stochastic variance as defined in
equations~\ref{eq:equation24} and~\ref{eq:equation76}.
We assume that the two measurements are related by

\begin{equation}
    \chi_r^2 = \frac{\text{Var} \{f(\bm{v}\, |\, \hat{c}_v) \}}
                    {\text{Var} \{f(\bm{v}\, |\, r)\}}
    \label{eq:equation26}
\end{equation}

Now, for a set of $n$ error-normalized observations the unbiased sample variance
is given by

\begin{equation}
    S_{n - 1}^{2} = \frac{1}{n - 1} \sum_{i=1}^{n}{(Z_i - \bar{Z})^2}
    \label{eq:equation28}
\end{equation}

where we set

\begin{equation}
    Z = \frac{y - \hat{y}}{\sigma_y}
    \label{eq:equation32}
\end{equation}

From~\citet{Mood1974} we know that the mean squared error is

\begin{align}
    \text{MSE}(S_{n-1}^{2}) &= \frac{1}{n}
        \left(\mu_4 - \frac{n-3}{n-1}\sigma_f^4
    \right) \nonumber \\
        &= \frac{1}{n}\left(\beta_2 + \frac{2n}{n-1}\right)\sigma_f^4
    \label{eq:equation81}
\end{align}

where $\mu_4$ is the fourth moment of the distribution, and $\beta_2$ is the
excess kurtosis, also given by

\begin{equation}
    \beta_2 = \frac{\mu_4}{\sigma_f^4} - 3
    \label{eq:equation84}
\end{equation}

Clearly, in this case $\text{MSE}(S_{n-1}^{2}) \approx \chi_r^2$ when
$n \gg 1$, and if we make the strong assumption that excess kurtosis is
independent of $\sigma_f$, then from equation~\ref{eq:equation81} we can write

\begin{equation}
    \chi_r^2 = \beta_s^2 \sigma_f^4
    \label{eq:equation85}
\end{equation}

where $\beta_s$ is a constant.
Since $\sigma_y$ (the error in the measured sample values) was supplied by the
user and is assumed to be correct, we allow $\sigma_f$ to represent an error
factor introduced by the weighting function.
A representative value for $\sigma_f$ can be derived by integrating
equation~\ref{eq:equation77} as

\begin{equation}
    \sigma_f = \int_{R^K}{w_{\delta}} =
        \left(\frac{\pi}{2}\right)^{K/2} \det(A)^{1/2}
    \label{eq:equation29}
\end{equation}

Therefore,

\begin{equation}
    \chi_r = \beta_s \frac{\pi}{2} \det(A)
    \label{eq:equation30}
\end{equation}

with $\chi_r = \sqrt{\chi_r^2}$.  $\beta_s$ can then be determined from a
suitable test reduction, and a new weighting function can be defined such that
$\chi_r \to 1$.
To accomplish this, and using our assumption that $\beta_s$ is constant, we set

\begin{equation}
    \det(A_{\text{new}}) = \det(A) / \chi_r
    \label{eq:equation31}
\end{equation}


\subsubsection{Test Reduction}\label{subsubsec:adaptive-test-reduction}

One fundamental assumption made during the adaptive weighting algorithm is that
the MSE is dependent on the excess kurtosis introduced by the
weighting function (see equation~\ref{eq:equation81}).
This approximation can only be relied upon when $n \gg 1$, and we therefore
need to select an initial kernel that can fulfill this requirement.
However, if the kernel is too large such that a low order polynomial could not
possibly represent the expected structure in the data, a misleadingly high
excess kurtosis would be reported due to an excess of outliers.

In astronomical observations (for which this algorithm was devised), the
Full-Width-Half-Maximum (FWHM) for the point spread function of the observing
system is usually known, at least approximately.
If so, we know the uncertainty in the positional measurements is given as

\begin{equation}
    \sigma_x = \frac{\text{FWHM}_x}{2 \sqrt{2 \ln 2}}
    \label{eq:equation34}
\end{equation}

To effectively nyquist sample the expected structures present in data retrieved
from an instrument with a Gaussian response function we set

\begin{align}
    \sigma_{\text{test}} &= \frac{\pi}{\sqrt{2 \ln{2}}}
                            \sigma_x \nonumber \\
                         &= \frac{\pi}{4\ln{2}} \text{FWHM}_x
    \label{eq:equation36}
\end{align}

We can then define the test distance weighting kernel as

\begin{equation}
    A_{\text{test}} = diag({2 \sigma_{\text{test}}^2})
                    = diag(\alpha_{\text{test}})
    \label{eq:equation33}
\end{equation}

There is one other important difference to the standard reduction:
the user supplied resampling points (coordinates at which fits are required)
are completely ignored, and instead replaced with the sample coordinates
themselves.
In other words, a fit is performed at each sample coordinate generated from
the values of all samples within its window region (see
figure~\ref{fig:adaptive-test_reduction}).

\begin{figure}[H]
  \begin{center}
  \includegraphics[width=0.75\linewidth]{images/adaptive_test_reduction.png}
  \caption{Left: A standard reduction on a single block.  Samples are shown
           as blue dots, and resampling points are shown as red crosses with
           the window region of a single point marked as a dashed red line.
           Right: A test reduction for determining the adaptive weighting
           kernels where resampling points are equal to the sample coordinates.}
  \label{fig:adaptive-test_reduction}
  \end{center}
\end{figure}

\subsubsection{Adaptive Weighting Kernel}
\label{subsubsec:adaptive-weighting-kernel}

A counter-intuitive consequence is that positional weighting kernels are
derived for each sample rather than at each resampling point.
This is extremely beneficial when repeatedly resampling the same sample set
onto different resampling points as the weighting kernels only need to be
calculated once.
Once generated, these kernels allow for almost real-time interactive analysis
of a large data set over a modestly sized region of interest.
Kernels are also self-consistent, dependent only on the samples themselves
rather than the resampling points defined by the user.

The final distance weighting applied to sample $i$ for a fit at point $m$ is

\begin{equation}
     w_(\delta, i, m) = exp \left(
              -(v_m - x_i)^T
               A_{\text{test}, i}^{-1}
               (v_m - x_i)
     \right)
     \label{eq:equation37}
\end{equation}

where $A_i$ is the adaptive weighting kernel (matrix of rank $K$), derived
for each sample $i$, $i = 1, \cdots, N$.
There are currently two adaptive weighting algorithms: ``scaled'' and
``shaped''.
The scaled algorithm applies a scaling factor to $\alpha_{test}$, expanding
or contracting the width of the kernel by an equal factor across all dimensions.
The second algorithm (shaped) is an extension of the first, available when
$K>1$, and allows the kernel to stretch and rotate about a given sample.

\paragraph{Adaptive Kernel Scaling}
\label{paragraph:adaptive-kernel-scaling}
During the test reduction, a fit is performed at each sample coordinate
$x_i$ derived from all samples within the window region $\Omega_i$.
We use a test smoothing parameter $\alpha_{\text{test}}$, defining the test
weighting kernel as

\begin{equation}
    A_{\text{test}} = diag(\alpha_{\text{test}})
    \label{eq:equation38}
\end{equation}

so that the scaled weighting kernel ($A_{scaled}$) will satisfy

\begin{equation}
    det(A_{\text{scaled, i}}) = \frac{det(A_{\text{test}})}{\chi_{r, i}}
    \label{eq:equation39}
\end{equation}

Since the scaled algorithm merely applies a global scaling factor to the
test diagonal kernel, we can write

\begin{align}
    det(A_{\text{scaled}}) &= \frac{det(A_{\text{test}})}{\chi_r}
    \nonumber \\
    \prod_{k=1}^{K}{\alpha_{\text{scaled}}}
        &= \frac{1}{\chi_r} \prod_{k=1}^{K}{\alpha_{\text{test}}}
    \nonumber \\
    \alpha_{\text{scaled}} &= \chi_r^{-\frac{1}{K}} \alpha_{\text{test}}
    \label{eq:equation40}
\end{align}


\subsubsection{Adaptive Kernel Shaping}
\label{subsubsec:adaptive-kernel-shaping}

The shaped adaptive weighting algorithm allows the overall kernel scale to
vary according to the scaled method described above, and allows the
principle axes of the kernel to stretch and rotate about each sample.
Note that this only has meaning in two dimensions or more ($K > 1$).
When attempted for $K=1$, the shaped kernel is equivalent to the scaled kernel.

To first order, variance on the fit may be propagated as

\begin{align}
    Var(y) &= J^T Var(x) J \nonumber \\
           &= V_x J^T I J \nonumber \\
           &= V_x J^T J
    \label{eq:equation35}
\end{align}

where $Var(x)$ is the variance-covariance matrix of the sample measurements.
We assume that $Var(x)$ is dimensionally independent, and may be represented
by the scalar $V_x$ as a function of measurement error only.
Here, $J$ is the Jacobian matrix of the fitting function given as

\begin{equation}
    J = \begin{bmatrix}
            \frac{\partial f_1}{\partial x_1} &
            \frac{\partial f_1}{\partial x_2} &
            \dots &
            \frac{\partial f_1}{\partial x_K} \\
            \frac{\partial f_2}{\partial x_1} &
            \frac{\partial f_2}{\partial x_2} &
            \dots &
            \frac{\partial f_2}{\partial x_K} \\
            \vdots & \vdots & \ddots & \vdots \\
            \frac{\partial f_n}{\partial x_1} &
            \frac{\partial f_n}{\partial x_2} &
            \dots &
            \frac{\partial f_n}{\partial x_K}
        \end{bmatrix}
    \label{eq:equation41}
\end{equation}

where $\frac{\partial f_i}{\partial x_k}$ is the partial derivative of the
function for sample $i$ with respect to dimension $k$.
Derivatives of the polynomial function are easily determined using a subset of
the $\Phi$ terms (see equation~\ref{eq:equation5}).
The set of polynomial terms defined in equation~\ref{eq:equation3} guarantees
that all necessary derivative terms are present so long as the polynomial order
is $\ge 1$.
For example, consider a second order polynomial fit in two dimensions

\begin{align}
    f(\bm{x}) &= c_1 + c_2 x_0 + c_3 x_0^2 + c_4 x_1 + c_5 x_0 x_1 + c_6 x_1^2
    \nonumber \\
    \frac{\partial f}{\partial x_0} &= c_2 + 2 c_3 x_0 + c_5 x_1 \nonumber \\
    \frac{\partial f}{\partial x_1} &= c_4 + c_5 x_0 + 2 c_6 x_1 \nonumber
\end{align}

In terms of $\Phi$, these become

\begin{align}
    f(\Phi) &= c_1 \Phi_1 + c_2 \Phi_2 + c_3 \Phi_3 + c_4 \Phi_4 +
               c_5 \Phi_5 + c_6 \Phi_6 \nonumber \\
    \frac{\partial f}{\partial x_0} &= c_2 \Phi_1 + 2 c_3 \Phi_2 + c_5 \Phi_4
    \nonumber \\
    \frac{\partial f}{\partial x_1} &= c_4 \Phi_1 + c_5 \Phi_2 + 2 c_6 \Phi_4
    \nonumber
\end{align}

Since all $\Phi$ terms are precalculated, calculating derivatives is a fast and
computationally cheap operation.
Once $f^{\prime}(x_j)\,\forall\, j \in \Omega_i$ has been evaluated, we create
the $(K, |\Omega_i|)$ matrix of weighted partial derivatives for $J^T J$ from
all samples in $\Omega_i$ by defining

\begin{equation}
    g_i = \frac{\partial f_j}{\partial x_k}\,,\,\forall\,
    k = \{1, 2, \cdots, K\}\,,\,\forall\,
    j \in \Omega_i
    \label{eq:equation49}
\end{equation}

and define the derivative mean-squared-cross-products (MSCP) as

\begin{equation}
    J^T J = \frac{1}{tr(W W^T)} g^T W W^T g
    \label{eq:equation50}
\end{equation}

where $W$ is the weight matrix defined in equation~\ref{eq:equation21}.
This results in the $(K, K)$ matrix

\begin{equation}
    J^T J =
        \begin{bmatrix}
            \frac{\partial \bar{f}}{\partial x_0}
            \frac{\partial \bar{f}}{\partial x_0} &
            \frac{\partial \bar{f}}{\partial x_0}
            \frac{\partial \bar{f}}{\partial x_1} &
            \dots &
            \frac{\partial \bar{f}}{\partial x_0}
            \frac{\partial \bar{f}}{\partial x_K} \\
            \frac{\partial \bar{f}}{\partial x_1}
            \frac{\partial \bar{f}}{\partial x_0} &
            \frac{\partial \bar{f}}{\partial x_1}
            \frac{\partial \bar{f}}{\partial x_1} &
            \dots &
            \frac{\partial \bar{f}}{\partial x_1}
            \frac{\partial \bar{f}}{\partial x_K} \\
            \vdots & \vdots & \ddots & \vdots \\
            \frac{\partial \bar{f}}{\partial x_K}
            \frac{\partial \bar{f}}{\partial x_0} &
            \frac{\partial \bar{f}}{\partial x_K}
            \frac{\partial \bar{f}}{\partial x_1} &
            \dots &
            \frac{\partial \bar{f}}{\partial x_K}
            \frac{\partial \bar{f}}{\partial x_K}
        \end{bmatrix}
    \label{eq:equation51}
\end{equation}

At this point we could define the normalized shape matrix as

\begin{equation}
    g = \frac{1}{\det(J^T J)^{1/K}}(J^T J)^{-1}
    \label{eq:equation52}
\end{equation}

where $\det(g) = 1$.
This normalization reflects that we are only interested in the shape of the
kernel, and that the overall scaling will still be handled according to
section~\ref{paragraph:adaptive-kernel-scaling}.
However, highly elongated kernels could result in poor fits for certain sample
distributions.
Additionally, if $\chi_r^2 \approx 1$ for the test fit, then we really do not
want to shape the kernel when it is already optimal.
Therefore, we provide a correction to the stretch of the kernel based on
number of factors.
Singular value decomposition (SVD) is used to factorize the normalized
$g$ matrix into

\begin{equation}
    U S V^T = \text{SVD}(g)
    \label{eq:equation53}
\end{equation}

Note that this will only be attempted if $g$ has full rank, and if
$\det(g) \neq 0$.
If not, determination of the kernel shape will be aborted, but the kernel will
still be scaled accordingly with $g = I$.

The singular values $S$ represent the relative magnitudes of the
partial derivatives in each dimension where $\prod_{k=1}^{K}{S_k} = 1$, and
the unitary ($X^T X = I$) $U$ and $V$ matrices apply rotation and/or reflection.
Therefore, the SVD of the gradient matrix can be thought of as an initial
rotation ($V$), followed by a stretch ($S$) and a final rotation ($U$).
To correct for stretch, we modify $g$ as follows:

\begin{equation}
    \bar{g} = U S^{\gamma} V^T
    \label{eq:equation54}
\end{equation}

where $\gamma$ is a real-valued ($-1 \leq \gamma \leq 1$) stretch correction
factor.
Note that for any value of $\gamma$, $\det(\bar{g}) = 1$.
i.e., $\bar{g}$ will always be normalized and not introduce any global scaling
factors into the kernel.
If $\gamma = 0$ the stretch along each dimensional axis will be equal, and
consequently, rotation is irrelevant so that the solution is equivalent to the
scaled solution.
If $\gamma = 1$ the shape remains unchanged such that $\bar{g} = g$.
Finally, if $\gamma = -1$, the kernel is effectively rotated so that it is
perpendicular to maximum gradient direction.
As $\gamma \rightarrow 0$ from either direction, the kernel becomes increasingly
symmetrical.
The effect of $\gamma$ on the shape kernel is shown in
figure~\ref{fig:stretch-correction}.

\begin{figure}[H]
  \begin{center}
  \includegraphics[width=\linewidth]{images/stretch_correction.png}
  \caption{The effect of $\gamma$ on the shape kernel.  The unaltered shape
           is returned when $\gamma=1$, symmetrical when $\gamma=0$, and
           stretch is reversed when $\gamma=-1$.}
  \label{fig:stretch-correction}
  \end{center}
\end{figure}

Now all that remains is the task of determining $\gamma$, which is dependent
on three factors:  $\chi_r^2$, the density profile of samples in the window
region ($\rho$), and the deviation of the center of the window from
the mean sample distribution ($\sigma_{\bar{x}}$).
The function $\gamma = f(\chi_r^2, \rho, \sigma_{\bar{x}})$ must satisfy

\begin{align}
    \lim_{\chi_r^2 \to \infty} \, &\gamma = 1 \nonumber \\
    \lim_{\chi_r^2 \to 0} \, &\gamma = -1 \nonumber \\
    \lim_{\rho \to 0} \, &\gamma = 0 \nonumber \\
    \lim_{\sigma_{\bar{x}} \to \infty} \, &\gamma = 0
    \label{eq:equation55}
\end{align}

One such function that can fulfill these requirements is the logistic function;
specifically, a modified version of the Richard's curve.

\begin{equation}
    \gamma = \frac{2}
             {\left(
                 {1 + (2^{exp(\sigma_{\bar{x}})} - 1)
                 exp(\rho (1 - \chi_r^2))}
             \right)^{1/exp(\sigma_{\bar{x}})}} - 1
    \label{eq:equation56}
\end{equation}

Some properties to note are that:

\begin{enumerate}
    \item{The sign of $\gamma$ is dependent on $\chi_r^2$ and is negative for
    $0 \leq \chi_r^2 < 1$, positive for $\chi_r^2 > 1$, and sets $\gamma=0$ at $\chi_r^2=1$.}
    \item{The growth rate increases with $\rho$.}
    \item{The asymptotes are at $\gamma=\pm1$, with the point of inflection
    (point of maximal growth) set by $\sigma_{\bar{x}}$.  The point of
    inflection is set at $\chi_r^2=1$ when $\sigma_{\bar{x}} = 0$, and
    approaches the upper asymptote as $\sigma_{\bar{x}}$ increases.}
\end{enumerate}

Figure~\ref{fig:sigmoid-function} shows how $\gamma$ varies with $\chi_r^2$,
and the effects of $\rho$ and $\sigma_{\bar{x}}$.

\begin{figure}[H]
  \begin{center}
  \includegraphics[width=\linewidth]{images/sigmoid_function.png}
  \caption{The stretch correction factor ($\gamma$) as a function of $\chi_r^2$.
           Left: The effect of relative density ($\rho$) on the shape of the
           function with fixed $\sigma_{\bar{x}}$.  Right: The effect of
           window deviation from the mean sample distribution
           ($\sigma_{\bar{x}}$) with fixed $\rho$.}
  \label{fig:sigmoid-function}
  \end{center}
\end{figure}

\paragraph{Relative Density}
\label{paragraph:relative-density}
The local relative density ($\rho$) of samples is required to calculate the
factor $\gamma$ in equation~\ref{eq:equation56} when determining the shaped
adaptive weighting kernel.
This is a measure such that $\rho=1$ when samples are uniformly distributed,
$0 \leq \rho < 1$ when a fit occurs in a local depression of the sample density,
and $\rho > 1$ when samples are clustered around the fit point.
The local relative density is given as:

\begin{equation}
    \rho = \frac{\rho(\text{measured})}{\rho(\text{uniform})}
    \label{eq:equation57}
\end{equation}

The expected uniform density of samples inside a window region with principle
axes given by $\omega_k$ in dimension $k$ is

\begin{equation}
    \rho(\text{uniform}) = N \frac{\Gamma \left( 1 + \frac{K}{2} \right)}
                              {\pi^{K/2} \prod_{k=1}^{K}{\omega_k}}
    \label{eq:equation58}
\end{equation}

The measured density inside the window is

\begin{equation}
    \rho(\text{measured}) = \frac
        {\sum_{i=1}^{N}{w_{\delta, i}}}
        {\int_{R^{\Omega}} w_{\delta}(\mathbf{\Delta x})}
    \label{eq:equation59}
\end{equation}

\begin{figure}[H]
  \begin{center}
  \includegraphics[width=\linewidth]{images/relative_density.png}
  \caption{Relative density values for different types of sample distributions.
           The center of each window region (red dashed circle) is marked with
           a red cross, and samples are marked as blue dots.  All plots were
           generated using $\omega = 1$ and $\alpha=0.3$.  Left: uniform
           sample distribution should give $\rho=1$.  In this case, the
           discrete nature of a small number of measurements leads to a slight
           deviation from the optimal value.  Middle: samples clustered near
           the center of the window yield $\chi_r^2 > 1$.  Right:
           $\chi_r^2 < 1$ when the window centeroccurs in a local minimum of
           the sample density.}
  \label{fig:relative_density}
  \end{center}
\end{figure}

where $\Delta x = \bm{v} - \bm{x}$ gives the relative position of samples to
the center of the window, and
$\int_{R^{\Omega}} w_{\delta}(\mathbf{\Delta x})$ is the integral of the
distance weighting function (given in equations~\ref{eq:equation25}
and~\ref{eq:equation37}) over the window region in $K$ dimensions.
This integral cannot be solved exactly, and thus, we resort to computationally
expensive numerical methods (although ``computationally expensive'' is relative
in this sense, and the operation takes only a small fraction of a second on
most laptops).
Fortunately, this only needs to be solved once as the test reduction uses a
single weighting function due to constant $\omega$ for all sample fits.
However, if one were to take an iterative approach to determining the optimal
kernel shape, this integral would need to be evaluated for each sample, as each
weighting function could be unique following an initial test reduction.
If this approach was taken, then it would be recommended to either disregard
density by setting $\rho=\text{constant}$ in equation~\ref{eq:equation56}, or
using the result obtained from the initial test reduction.
Examples of local density values for several types of sample distribution can
be seen in figure~\ref{fig:relative_density}.

\paragraph{Window Offset Deviation}
\label{paragraph:window-offset-deviation}

The window offset deviation ($\sigma_{\bar{x}}$) that is applied in
equation~\ref{eq:equation56} is given as

\begin{equation}
    \sigma_{\bar{x}}^2 = (\bm{v} - \bar{\bm{x}})^{T}
        \Sigma_{x}^{-1} (\bm{v} - \bar{\bm{x}})
    \label{eq:equation60}
\end{equation}

for a distribution of $N$ samples with coordinates $\bm{x}$ with mean position
$\bar{\bm{x}}$.
The window region is centered on $\bm{v}$ where the covariance of the sample
distribution between dimensions $i$ and $j$ is given as

\begin{equation}
        \Sigma_{x_{i,j}} = \frac{1}{N - M} \sum_{n=1}^{N}
            {(\bar{\bm{x}}_{n, i} - \bar{\bm{x}}_i)
             (\bm{x}_{n, j} - \bar{\bm{x}}_j)}
    \label{eq:equation61}
\end{equation}

Here, $M$ are the number of degrees of freedom lost when determining the
center of the sample distribution $\bar{\bm{x}}$.  $M$ is typically set to 1
when a simple mean operation is performed.
Figure~\ref{fig:distribution_error_2d}, in addition to displaying edge
rejection thresholds, also shows lines of constant $\sigma_{\bar{x}}$ for
a normally distributed set of samples.

\paragraph{Shaped Adaptive Kernel}
\label{paragraph:shaped-adaptive-kernel}

Now all parameters required for the derivation of the shaped adaptive weight
matrix have been derived, all that remains is to define the kernel itself.

\begin{equation}
    A_{\text{shaped}} = \left(
        \frac{\det(A_{\text{test}})}{\chi_r}
    \right)^{1 / K} \hat{g}
    \label{eq:equation62}
\end{equation}

Note that once again, $\det(A_{\text{test}}) /\det(A_{\text{shaped}}) = \chi_r$.
Unlike the scaled solution, the shape of the kernel depends only on
$\hat{g}$ and does not propagate the shape of $A_{\text{test}}$.
Therefore, since all information on the shape of $A_{\text{test}}$ is lost,
it is recommended that the user sets the ratio of $\omega / \sigma_x$ to be
equal for all dimensions in which adaptive weighting is enabled
(see section~\ref{paragraph:partial-adaptive-weighting}), or to otherwise
use the scaled weighting kernel.

As a brief graphical example, the scaled and shaped kernels are
generated from an image (samples are pixels) and then plotted for comparison
in figure~\ref{fig:adaptive-kernels}.
Note that the kernels are large in smooth regions, and small in regions
containing complex structure and edges.
The area of a shaped and scaled kernel on a single sample will always
be equal.
However, the shaped kernels are truncated in the direction of the gradient,
and will preserve edges to a greater degree.

\paragraph{Partial Adaptive Weighting for Select Dimensions.}
\label{paragraph:partial-adaptive-weighting}

The size and/or shape of the kernel may be fixed in certain dimensions and
allowed to vary in others.
To accomplish this, the number of fixed and adaptive enabled dimensions as
set to $K_{\text{fix}}$ and $K_{\text{adapt}}$ where
$K_{\text{fix}} + K_{\text{adapt}} = K$, and the sets of dimensions
allocated to each are defined as $k_{\text{fix}}$ and $k_{\text{adapt}}$.
For the test reduction, the smoothing parameter is set to

\begin{equation}
    \alpha_{\text{test, k}} = \begin{cases}
        \alpha_k, & \textrm{if } k \in k_\text{fix} \\
        \alpha_{\text{test}, k}, & \textrm{if } k \in k_\text{adapt}
    \end{cases}
    \label{eq:equation63}
\end{equation}

\begin{figure}[H]
  \begin{center}
  \includegraphics[width=0.75\linewidth]{images/adaptive_kernels.png}
  \caption{Comparison of the scaled (solid green) and shaped (dashed
           red) kernels.  The upper-left image was used as an input for the
           test reduction where each pixel used as a sample.  The bottom-left
           image contains a black-dotted kernel, representing the test kernel
           with smoothing parameter $\alpha_{test}$.  Kernels are relatively
           large in smooth regions and smaller in regions containing structure.
           Note that the shaped kernels are stretched perpendicular to the
           principle gradient direction, and will preserve edges well in
           subsequent reductions.  The data for this image was taken
           from skimage.data.camera (see \cite{skimage2014}).}
  \label{fig:adaptive-kernels}
  \end{center}
\end{figure}

All calculations are performed as described in
sections~\ref{paragraph:adaptive-kernel-scaling}
and~\ref{paragraph:shaped-adaptive-kernel} with a few important changes:

The final scaled adaptive kernel is given as

\begin{equation}
    A_{\text{scaled}, (i,j)} = \begin{cases}
        \alpha_i, & \textrm{if } i=j \textrm{ and } i \in k_{\text{fix}} \\
        0,& \textrm{if } i \neq j \\
        \chi_r^{-1/K_{\text{adapt}}} \alpha_{\text{test}},&
        \textrm{if } i=j \textrm{ and } i \in k_{\text{adapt}}
    \end{cases}
    \label{eq:equation65}
\end{equation}

For the shaped kernel, a modification is made to $\hat{g}^2$ (following the
stretch correction) so that

\begin{equation}
    {\hat{g}}_{i, j} = \begin{cases}
        1,& \textrm{if } i = j \textrm{ and } i \in k_{\text{fix}} \\
        0,& \textrm{if } i \neq j \textrm{ and } i \lor j \in k_{\text{fix}}
    \end{cases}
    \label{eq:equation66}
\end{equation}

and the final shaped adaptive kernel is given as

\begin{equation}
    A_{\text{shaped}, (i,j)} = \begin{cases}
        \alpha_i, & \textrm{if } i=j \textrm{ and } i \in k_{\text{fix}} \\
        0,& \textrm{if } i \neq j \textrm{ and } i \lor j \in k_{\text{fix}} \\
        \left(
            \frac{\prod_{k \in k_{\text{adapt}}}{\alpha_{\text{test},k}}}
                 {\det(\hat{g}) \chi_r}
        \right)^{\frac{1}{K_{\text{adapt}}}} {\hat{g}}_{i, j},
        & \textrm{if } i \land j \in k_{\text{adapt}}
    \end{cases}
    \label{eq:equation67}
\end{equation}

Consideration must be given to the fact that for high $K$ reductions where
$K_{\text{adapt} \to 1}$, the change in size of a kernel along an
adaptive-enabled dimension becomes increasingly severe as it must accommodate
the necessary change in overall kernel volume based on $\chi_r^2$.
The effect of fixing certain dimensions in the adaptive weighting calculation
can be seen in figure~\ref{fig:fix-adaptive}.

\begin{figure}[H]
  \begin{center}
  \includegraphics[width=0.75\linewidth]{images/fix_adaptive.png}
  \caption{Fixing the smoothing parameter in a single dimension of the
  shaped adaptive kernel.  In this example, $\chi_r^2=4$ and an arbitrary
  derivative MSCP of $[[3, 1], [1, 1]]$ was supplied with minimal stretch
  correction ($\rho \gg 1$).  A single line of constant weighting is drawn for
  each kernel, with $A_{\text{test}}$ shown as a dotted black circle with
  $\alpha_{\text{test}, 0} = \alpha_{\text{test}, 1}$ (circular).  In all
  cases, the ratio of kernel determinants is equal to
  ($\det(A_{\text{test}}) / \det(A_{\text{fix}}) = \sqrt{\chi_r^2}$).}
  \label{fig:fix-adaptive}
  \end{center}
\end{figure}


\section{Code Architecture}\label{sec:code-architecture}
\subsection{Language and packages}
\label{subsec:packages}

The resampling code is written in Python \footnote{Available from the
\href{http://www.python.org}{Python Software Foundation}}
(see~\citet{Python1995}) version 3.
Python is a dynamic open source programming language supported by a vast
ecosystem of specialized packages of which we rely on only a few:
Data such as the sample values, errors, $\Phi$, etc., are stored as Numpy arrays
(\citet{Numpy}) and processed using JIT (Just-In-Time) compiled Numba
functions (\citet{Numba}).
Numba was instrumental in performing multidimensional resampling in a
reasonable time frame since Python is notoriously slow when compared to
other languages, and allows for operations to be performed on a time scale
comparable with the C language.
The balltree algorithm (\citet{Balltree}) from the Scikit-learn package
(\citet{sklearn}) is used to quickly determine which samples are inside the
window region of a resampling point (see
figure~\ref{fig:single_thread_reduction}).
Finally, parallel processing is handled by the Joblib
\footnote{Available from \href{https://joblib.readthedocs.io/en/latest/}
{https://joblib.readthedocs.io/en/latest/}}.
The resampling code exists as a submodule of the toolkit module in the SOFIA
data reduction pipeline package (REDUX) that is not yet publicly available.
However, efforts are being made to release the REDUX package within the year.

\subsection{Classes and Structure}
\label{subsec:class-structure}

Figure~\ref{fig:software-flow} gives a high level representation of the
various classes and process flow.
The first step is to create an instance of the Resampler class, initialized
with some parameters that must include the window defining the
''local'' region ($\omega$), the polynomial fit order, and the data we wish to
resample.
The user may interact with the Resampler through some type of interface.
Nearly all examples in this paper were generated from the Python command line
and Jupyter notebooks, although some FIFI-LS images were created using the
SOFIA data reduction pipeline (REDUX).

During the instantiation of the Resampler object, the sample coordinates ($X$)
are used to initialize a Tree object that allocates samples into blocks and
their associated neighborhoods as described in
section~\ref{subsec:search-problem} as a function of the window region $\omega$.
The $\Phi(x)$ terms (see
section~\ref{subsec:polynomial-representation-of-k-dimensional-data}) are also
created at this stage and stored as a Tree attribute.

Once the Resampler object has been created, a reduction may be started by
supplying a set of reduction parameters (such as weighting schemes) and a set
of coordinates (resampling points) containing the desired locations for the
fitted output values.
Another Tree object is created by the reduction manager from the resampling
point coordinates ($V$), allocating blocks and creating a set of $\Phi(V)$
terms.

Each intersection of a reduction Tree block with a sample Tree neighborhood is
than passed in parallel from the reduction manager to the main engine of the
resampling algorithm (the box labelled ``Fit'' in
figure~\ref{fig:software-flow}).
For a single block of resampling points, the sample sets $\Omega$ for each
point are derived simultaneously from the sample neighborhood using the
balltree algorithm.
A fit at each point is processed in series before being passed back to the
reduction manager where they are aggregated.
Once all parallel reductions are complete, the results become available to the
Resampler where they are either used to update the reduction parameters when
determining an adaptive solution (resulting in a second reduction) or passed
back to the user.

\begin{figure}[H]
  \begin{center}
  \includegraphics[width=\linewidth]{images/software_flow.png}
  \caption{Main software objects and process flow for the full resampling
           algorithm.  Data and variables are denoted by a green open circle.
           Notes and actions are marked by a filled black circle.  Aggregate
           objects (can exist independently of the parent object) are marked
           with open diamond heads, while composite objects (cannot exist
           without parent object) are marked with a filled diamond head.}
  \label{fig:software-flow}
  \end{center}
\end{figure}




\section{Applications}\label{sec:applications}
\subsection{2-Dimensional Resampling of Regular Data}
\label{subsec:example-2d}

The following example uses a ($128 \times 128$ pixels) subset of the ``camera''
test data set from the Python scikit-image package (\citet{skimage2014}),
already displayed in figure~\ref{fig:adaptive-kernels} to show the effects of
the resampling algorithm applied to regularly spaced 2-D samples onto a finer
grid where $\Delta x = 0.2$ pixels.
The FWHM of the observing device was estimated to be 1 pixel (by eye), and the
error was estimated from the standard deviation of the blank sky region on the
upper-right corner of the image ($\approx 0.7\%$ of the maximum sample value).
The window radius is set to 12 pixels resulting in a median of 452 samples
per 3rd order polynomial fit.

Figure~\ref{fig:camera-fits} shows the sample image, and fits on a finer grid
using standard distance weighting with
$\alpha = 2 \left(\text{FWHM} / 2\sqrt{2 \ln 2}\right)^2$ and the results of
the scaled and shaped adaptive distance weighting algorithm.
In addition, different edge and order rejection algorithms were applied, the
effects of which can be seen on the borders of each fitted image.
The standard distance weighting algorithm rejected fits using the bounded
order requirement, while the adaptive fits allowed for extrapolation in cases
where enough unique samples are present.
However, the ``scaled'' solution applied an edge rejection threshold of
$\beta_{\text{edge}} = 1$.
In this specific case, edge and order rejection options are applied for the sake
of example only.
Good fits are possible due to regularly spaced sampling, well-behaved data,
and that we are not attempting any fits outside of the bounds of the sample
region.

\begin{figure}[H]
  \begin{center}
  \includegraphics[width=0.9\linewidth]{images/camera_fits.png}
  \caption{Resampling an image (upper-left) onto a finer grid using the
           standard distance weighting algorithm (upper-left), the ``scaled''
           adaptive weighting algorithm (lower-left), and the ``shaped''
           adaptive weighting algorithm (lower-right).  The red and green
           squares on the upper-left image mark regions shown in
           figure~\ref{fig:camera-samples}.}
  \label{fig:camera-fits}
  \end{center}
\end{figure}

At first glance, the standard and scaled solutions appear fairly similar to the
original image, while a strong reduction in the pixelation effects can be
seen in the shaped image.
The similarity of the standard and scaled solutions is cosmetic only,
and show notable differences when viewed on a smaller local scale.
Figure~\ref{fig:camera-samples} show enhanced sections of the fit on a
smaller scale which are marked for reference on the upper left image of
figure~\ref{fig:camera-fits}.

\begin{figure}[H]
  \begin{center}
  \includegraphics[width=0.8\linewidth]{images/camera_samples.png}
  \caption{Subsections of the sample image and fits marked by the regions in
           figure~\ref{fig:camera-fits}.  The upper row (``Feature''),
           contains an area with strong features (green box in
           figure~\ref{fig:camera-fits}), while the lower row (``Sky'') shows
           a relatively smooth portion of the sample data (red box in
           figure~\ref{fig:camera-fits}).  The columns from left to right
           display the sample image, the standard distance weighted fit, the
           scaled adaptive fit, and the shaped adaptive fit.}
  \label{fig:camera-samples}
  \end{center}
\end{figure}

\begin{figure}[H]
  \begin{center}
  \includegraphics[width=0.8\linewidth]{images/camera_chi2.png}
  \caption{Histograms of $\log_{10}(\chi_r^2)$ for each of the distance
           weighting algorithms.  The normalized count density is given as
           $\rho(N) = N_i / \left(\Delta P \sum_{i}{N_i}\right)$ where
           $\Delta P$ is the spacing between bins.}
  \label{fig:camera-chi2}
  \end{center}
\end{figure}

Finally, figure~\ref{fig:camera-chi2} displays histograms of the $\chi_r^2$
distribution for all fits over the entire resampled images (409,600 pixels).
On a ${\log}_{10}$ scale where $P = {\log}_{10}(\chi_r^2)$, the distribution can
be approximated by a normal distribution
$\mathcal{N}(\text{mean}(P), \text{variance}(P))$.
For our example fits $\mathcal{N}_{\text{standard}} = (-3.126, 0.905)$,
$\mathcal{N}_{\text{scaled}} = (0.198, 0.371)$, and
$\mathcal{N}_{\text{shaped}} = (0.092, 0.442)$.
In other words, the adaptive weighting algorithms produced fits where $\chi_r^2$
is more closely centered around 1 with smaller variance.

\subsection{3-Dimensional Resampling of Irregular Data}
\label{subsec:example-fifi-ls}

The following examples apply the new resampling algorithm to real data taken
with the FIFI-LS instrument on board SOFIA (see
section~\ref{subsec:motivation}).

\subsubsection{M82: A comparison of weighting schemes on a high SNR source}

Observations of M82 were taken over multiple SOFIA flights using the long
wavelength spectrometer.
Data are irregularly spaced in cross-elevation, elevation, and wavelength
($x, y, \lambda$) centered on $\lambda=157.694\mu m$, and J2000 coordinates (09h55m55.38s
+69d40m53.4s) with $x$ and $y$ in units of arcseconds.
The average spatial FWHM of the observations is
$\text{fwhm}_x = \text{fwhm}_y = 15.8\arcsec$, and the average spectral FWHM is
$\text{fwhm}_{\lambda} = 0.06689507\mu m$.
Histograms of the sample measurement values and errors are shown in
figure~\ref{fig:fifi-M82-sample-errors}.

The sample distribution is shown in
figure~\ref{fig:fifi-M82-sample-distribution}.
Note that the spatial (x, y) distribution consists of a sparsely sampled region
near the center bounded by two more densely sampled areas above and below, as
shown on the left image.
It can be seen from the right-most image that the wavelength sampling interval
For a single cluster of spatial coordinates shown on the left image as a single
blue dot, the median $\lambda$ sampling interval is $\Delta \lambda = 0.003482 \mu m \approx 19
\text{ samples}/\text{fwhm}_{\lambda}$.

\begin{figure}[H]
  \begin{center}
  \includegraphics[width=\linewidth]{images/fifi_M82_sample_errors.png}
  \caption{Histograms of the sample measurement values (left) and
  $1\sigma$ measurement errors (right).}
  \label{fig:fifi-M82-sample-errors}
  \end{center}
\end{figure}

\begin{figure}[H]
  \begin{center}
  \includegraphics[width=\linewidth]
      {images/fifi_M82_sample_distribution.png}
  \caption{Left: Spatial sample distribution in x (cross-elevation) and y
           (elevation) coordinates.  Right: Sample distribution displayed in
           x, y, and $\lambda$ (wavelength).}
  \label{fig:fifi-M82-sample-distribution}
  \end{center}
\end{figure}

Resampling was conducted to determine solutions for the standard, scaled, and
shaped distance weighting algorithms on a regular spaced ($x, y$)
($30 \times 72$) grid ($M=2,160$) with  spacing $\Delta x = \Delta y = 3\arcsec$
at constant $\lambda=157.83418\mu m$.
The window region was set to $\omega_{x,y}=\text{fwhm}_{x,y} \times 3$, and
$\omega_{\lambda} =\text{fwhm}_{\lambda} / 2$ resulting in a median of
4,200 samples per second order polynomial fit in a sample space containing
$N=147,200$ samples.
Fits were rejected according to the bounded order rejection algorithm, along
with an edge rejection threshold of $\beta_{\text{edge}} = 0.7$ in the spatial
dimensions, and 0.5 in $\lambda$.
For the standard distance weighting reduction, the smoothing parameter was
set to $\alpha_{x,y} = \text{fwhm}_{x,y}$, and
$\alpha_{\lambda} = \text{fwhm}_{\lambda} / 4$.
Adaptive weighting (scaled and shaped) was only applied to the spatial
dimensions, and fixed at $\alpha_{\lambda}$ in the wavelength dimension.

Figure~\ref{fig:fifi-M82-image} displays all three fits on the same scale,
whereas figure~\ref{fig:fifi-M82-log-image} displays each fit normalized to
within $10^{-3} \to 1$ on a log scale in order to better represent the
dynamic range.

Clear discontinuities can be seen from a visual inspection of the standard
distance weighting reduction in figure~\ref{fig:fifi-M82-image} in areas
where spatial sampling density appears to transition from one value to
another as shown in the left image of
figure~\ref{fig:fifi-M82-sample-distribution}.
However, both adaptive weighting algorithms appear smooth and continuous
in these density transition regions.

Reduction times on a MacBook Pro with a 2.5 GHz Intel Core i7 processor using
16 GB of memory is 0.6 seconds using the standard weighting algorithm, and
22 and 20 seconds for the ``shaped'' and ``scaled'' weighting algorithms,
respectively.

This factor in increased reduction time represents the fact that kernels are
calculated for each sample (in the neighborhoods of all resampling points)
rather than at each resampling point, and that $M \ll N$.
Generally, when $M \approx N$, the reduction time for an adaptive weighting
algorithm is roughly 2 to 3 times that of the standard weighting algorithm.

\begin{figure}[H]
  \begin{center}
  \includegraphics[width=0.8\linewidth]{images/fifi_M82_image.png}
  \caption{Images generated for $\lambda=157.83418\mu m$ with each available
           weighting algorithm.  Black dotted lines indicate regions centered
           on constant $x=4.06\arcsec$ and $y=1.43\arcsec$ of
           width $\text{fwhm}_{x,y}$.  The samples plotted in
           figure~\ref{fig:fifi-M82-cross-sections} reside in these regions,
           with the cross marking a line of constant $\lambda=157.83418\mu m$ through
           the image.}
  \label{fig:fifi-M82-image}
  \end{center}
\end{figure}

\begin{figure}[H]
  \begin{center}
  \includegraphics[width=0.8\linewidth]{images/fifi_M82_log_image.png}
  \caption{Images from figure~\ref{fig:fifi-M82-image} displayed on a log
           scale.}
  \label{fig:fifi-M82-log-image}
  \end{center}
\end{figure}

\newpage

Figure~\ref{fig:fifi-M82-cross-sections} displays fits in which the coordinates
for two dimensions are held fixed, and resampling occurs along the remaining
($x$, $y$, or $\lambda$) dimension as marked in the left image of
figure~\ref{fig:fifi-M82-image} (over the peak).
The scale of the Gaussian standard deviation of the weighting function is
plotted for each dimension.
In those dimensions in which adaptive weighting is enabled ($x, y$), the
minimum and maximum standard deviations of the ``scaled'' weighting kernels are
also drawn.
For reference, all samples within $\pm \text{fwhm} / 2$ are plotted with color
representing the Mahalanobis distance of a sample from the resampling point
in the fixed dimensions.
Color indicates the sample coordinate deviation of coordinates in the two
fixed dimensions with respect to the resampling point coordinate.
i.e., for the top plot in figure~\ref{fig:fifi-M82-cross-sections} along $x$,
where $y=1.43\arcsec$ and $\lambda=157.83418 \mu m$

\begin{equation}
    \Delta \sigma(y, \lambda) =
        \sqrt{\left( \frac{y - 1.43}{\sigma_y} \right)^2 +
              \left( \frac{\lambda - 157.83418}{\sigma_{\lambda}} \right)^2
        }
    \nonumber \label{eq:equation68}
\end{equation}

\begin{figure}[H]
  \begin{center}
  \includegraphics[width=0.75\linewidth]{images/fifi_M82_cross_sections.png}
  \caption{Cross sections of the fit along each dimension for each weighting
           scheme.  Top: A fit along the $x$ dimension, with $y$ and $\lambda$
           held constant.  Center: A fit along the $y$ dimension, with $x$ and
           $\lambda$ held constant.  Bottom: A fit along the $\lambda$
           dimension, with $x$ and $y$ held constant.  The intersection of all
           fixed dimensions is marked by a cross in the left image of
           figure~\ref{fig:fifi-M82-image}, and samples within the marked
           regions are plotted for reference.  Samples are color coded according
           to deviation of each sample to the resampling point in dimensions
           that are held constant.}
  \label{fig:fifi-M82-cross-sections}
  \end{center}
\end{figure}

\newpage

Figure~\ref{fig:fifi-M82-rchi2} shows the resulting $\chi_r^2$ distribution
for the fit along each dimension.
In all cases, $\chi_r^2 > 1$ indicating a poor fit.
However, this is the result of attempting a fit on data samples that cannot be
modelled within the expected $1\sigma$ noise limits supplied with the data.
From figure~\ref{fig:fifi-M82-sample-errors}, the typical supplied sample
measurement errors are $\epsilon=10 Jy$.
A quick look at figure~\ref{fig:fifi-M82-cross-sections} shows that the range
of values at a single coordinate can vary by approximately $\pm 200 Jy$ between
samples that occupy close spatial/spectral coordinates.

\begin{figure}[H]
  \begin{center}
  \includegraphics[width=0.9\linewidth]{images/fifi_M82_rchi2.png}
  \caption{Histograms of the $\chi_r^2$ values for each cross-section in $x$,
  $y$, and $\lambda$ at 5,000 regularly spaced resampling points
  along each dimension into 100 bins.}
  \label{fig:fifi-M82-rchi2}
  \end{center}
\end{figure}

For reference, a fit generated using standard distance weighting with
$\alpha = [\alpha_x, \alpha_y, \alpha_{\lambda}] / 12$ is shown in
figure~\ref{fig:fifi-M82-rchi2-equal-1} resulting in a $\chi_r^2$ approximately
centered around 1.
The resulting fit is unusable with no discernible structure and must be placed
on a $\log$ scale for visualization.

\begin{figure}[H]
  \begin{center}
  \includegraphics[width=\linewidth]{images/fifi_M82_chi2_equal_1.png}
  \caption{The result of fitting M82 sample data to $\approx \chi_r^2=1$.
           Left: Histogram of $\log_{10}(\chi_r^2)$ into 50 bins.
           Right: The image slice (also shown in
           figure~\ref{fig:fifi-M82-image}) when fit approximating
           $\chi_r^2=1$.  Colors represent flux (Jy) on a $\log_{10}$ scale
           of the absolute value.}
  \label{fig:fifi-M82-rchi2-equal-1}
  \end{center}
\end{figure}

\clearpage
\newpage

\subsubsection{30 Doradus: A comparison of simultaneous vs. sequenced
               dimensional resampling}

The 30 Doradus data set is intended to highlight some differences between the
old and new FIFI-LS data reduction pipelines, and the advantages of fitting
multiple dimensions simultaneously.
The initial pipeline sequentially resampled the data in wavelength (1 dimension)
followed by spatial resampling in two dimensions onto a regular grid.

Our example data set contains multiple observations of 30 Doradus
totalling 481,600 (x, y, $\lambda$) FIFI-LS sample measurements which are
resampled onto a ($327 \times 356 \times 50$) grid.
Unlike the previous M82 example, the sampling density is relatively uniform
near the area of interest, but the SNR ratio is much lower.
The sample distribution is displayed in
figure~\ref{fig:fifi-30DOR-sample-distribution}, and histograms of sample
measurement values and errors are shown in
figure~\ref{fig:fifi-30DOR-sample-errors}.

The central wavelength of the observations was at $\lambda = 51.87619 \mu m$
with $\text{FWHM}_{\lambda} = 0.056389 \mu m$, and
$\text{FWHM}_{x,y} = 6.2\arcsec$.
For both resampling algorithms, the window radius in the spectral dimension
is set to $\omega_{\lambda} = \text{FWHM}_{\lambda} / 2$.
When resampling in 3-dimensions simultaneously, the spatial window radius
is set to $\omega_{x,y} = 3 \times \text{FWHM}_{x, y}$.
Spectral resampling drastically reduces the number of samples available for
spatial resampling.
To allow for a smooth fit, the standard practice using the old pipeline was to
increase the spatial window radius to
$\omega_{x,y} = 6 \times \text{FWHM}_{x, y}$.
In both cases, a second order polynomial fit is generated along the spectral and
spatial dimensions using the standard distance weighting algorithm with a
smoothing parameter of
$\alpha_{\lambda} = \text{FWHM}_{\lambda} / 4$ and
$\alpha_{x,y} = 2 \times \text{FWHM}_{x,y}$.

\begin{figure}[H]
  \begin{center}
  \includegraphics[width=\linewidth]{images/fifi_30DOR_sample_distribution.png}
  \caption{Left: The spatial distribution of samples in x (cross-elevation),
           and y (elevation).  Right: The full 3D sampling distribution in
           x, y, and $\lambda$ (wavelength).}
  \label{fig:fifi-30DOR-sample-distribution}
  \end{center}
\end{figure}

\begin{figure}[H]
  \begin{center}
  \includegraphics[width=0.75\linewidth]{images/fifi_30DOR_sample_errors.png}
  \caption{Left: Histogram of the sample measurement values.  Right:
           Histogram of the associated $1\sigma$ measurement errors.}
  \label{fig:fifi-30DOR-sample-errors}
  \end{center}
\end{figure}


Many properties of the reduction are hard to compare due to major differences in
the way data are handled (for example, number of samples per fit or $\chi_r^2$).
We do, however, present comparisons of both a spatial and spectral slice that
intersect at the point of greatest emission.

Figure~\ref{fig:fifi-30DOR-image-comparison} displays a spatial slice at
$\lambda = 51.94550 \mu m$, and a spectral profile is shown in
figure~\ref{fig:fifi-30DOR-line-comparison} at (RA, Dec) =
(5h38m45.9046s, -69d05m06.2397s) which is also marked as a black cross on the
left image of figure~\ref{fig:fifi-30DOR-image-comparison}.

\begin{figure}[H]
  \begin{center}
  \includegraphics[width=\linewidth]{images/fifi_30DOR_image_comparison.png}
  \caption{Spatial slices at $\lambda=51.94550\mu m$.  Left: The result of
           first resampling in wavelength followed by the 2-dimensional spatial
           interpolation.
           Right: resampling in all dimensions simultaneously.
           The spatial position of the spectral profile shown in
           figure~\ref{fig:fifi-30DOR-line-comparison} is marked as a black
           cross on the left image.  The red circle encloses a prominent
           artifact.}
  \label{fig:fifi-30DOR-image-comparison}
  \end{center}
\end{figure}

\begin{figure}[H]
  \begin{center}
  \includegraphics[width=\linewidth]{images/fifi_30DOR_line_comparison.png}
  \caption{The spectral profile at RA = 5h38m45.9046s, Dec = -69d05m06.2397s.
           The red line shows the results of sequential resampling, while the
           blue line shows the results of directly resampling onto a
           3-dimensional grid.}
  \label{fig:fifi-30DOR-line-comparison}
  \end{center}
\end{figure}

It can easily be seen that resampling in all dimensions simultaneously using
the new FIFI-LS pipeline has produced a smoother image with fewer artifacts
than the older version, although peak fluxes are relatively equal.
Additionally, the spectral profile produced by the 3-dimensional fit is smooth
and continuous, whereas discontinuities are clearly visible when resampling
occurs sequentially over spectral and spatial dimensions.
In fact, the strong artifact shown by the red circle in
figure~\ref{fig:fifi-30DOR-image-comparison} is not consistent with slices
at neighboring wavelengths, and is colinear with many other artifacts that
are on the boundary of a change in sampling density as shown in
figure~\ref{fig:fifi-30DOR-sample-distribution}, indicating that it is not a
valid structure.


\section{Future Work}\label{sec:future-work}

This paper is primarily concerned with local polynomial regression, but could
easily be extended to other families of functions or filters.
For example, Gaussian, Lorentzian, Voigt, trigonometric, median etc.
Modified versions of the algorithm already exist for the weighted median
or mean of a local region, but do not yet implement adaptive weighting.

Another improvement that could be made would be to allow for variable window
regions.
Currently, the user must explicitly select a single set of window dimensions
($\omega$) applied around each resampling point.
In practice, this should be set to some minimum value to allow sufficient
samples from which to derive a fit, but also be large enough to account for
expansion of the adaptive weighting kernel.
Since computational complexity is $\propto \mathcal{O}(|\Omega|)$, the window
dimensions can have a significant effect on total reduction time.
Given that selecting a small window region can invalidate certain fits, the
user may be forced to choose a large window region that rectifies this
situation effecting only a few points at the cost of increased processing time.

\section{Conclusion}\label{sec:conclusion}
We have introduced a new procedure by which simultaneous multi-dimensional
interpolation can be carried out.
Consideration has been given to the distribution of irregularly sampled data,
and with appropriate choices, spurious fits can be reduced or eliminated.
An optional method is available to determine adaptive weighting factors in a
single step, approximating $\chi_r^2 \to 1$ on subsequent reductions.
These methods have been applied to resample three-dimensional irregular data
observed with the FIFI-LS instrument onto a regular data cube.


\acknowledgments
This work was supported by funds granted to the Universities Space Research
Association, Inc.\ (USRA) through the NASA/DLR Stratospheric Observatory for
Infrared Astronomy (SOFIA) program.
SOFIA is jointly operated by USRA under NASA contract NNA17BF53C and the
Deutsches SOFIA Institut (DSI) under DLR contract 50 OK 0901 to the University
of Stuttgart.
%%%%%%%%%%%%%%%%%%%%%%%%%%%%%%%%%%%%%%%%%%%%%%%%%%%%%%%%%%%%

\begin{thebibliography}{}

\bibitem[Colditz et al.(2018)]{Colditz18}
Colditz S.\ et al.,\ 2018,
Spectral and Spatial Characterization and Calibration of FIFI-LS -- The Field
Imaging Spectrometer on SOFIA,
JAI, Volume 7, 1840004-(12)556

\bibitem[Fan and Gijbels(1996)]{Fan96a}
Fan, J.,\ \& Gijbels, I.\ 1996,
Local polynomial modelling and its applications,
Monographs on Statistics and Applied Probability 66 (Vol.\ 66).\ CRC Press.

\bibitem[Fan et al.(1996)]{Fan96b}
Fan, J.,\ Gijbels, I.,\ Hu, T. C.,\ \& Huang, L. S.\ 1996,
A study of variable bandwidth selection for local polynomial regression.
Statistica Sinica,\ 113--127.

\bibitem[Fischer et al.(2018)]{Fischer18}
Fischer, C.\ et al.\ 2018,
FIFI-LS: The Field-Imaging Far-Infrared Line Spectrometer on SOFIA .
JAI, 7, 1840003--556.

\bibitem[Gelman and Imbens(2019)]{Gelman19}
Gelman, A.,\ \& Imbens, G.\ 2019,
''Why high-order polynomials should not be used in regression discontinuity
designs,''
Journal of Business \& Economic Statistics,\ 37(3), 447--456.

\bibitem[Gu et al.(2015)]{Gu15}
Gu, J.,\ Li, Q.,\ \& Yang, J. C.\ 2015,
''Multivariate local polynomial kernel estimators: Leading bias and asymptotic
distribution,''
Econometric Reviews,\ 34(6--10), 979--1010.

\bibitem[Masry(1996)]{Masry1996}
Masry, E.,\ 1996,
''Multivariate local polynomial regression for time series: uniform strong
consistency and rates,''
Journal of Time Series Analysis,\ 17(6),\ 571--599.

\bibitem[Mood(1974)]{Mood1974}
Mood, A.,\ Graybill, F.,\ \& Boes, D.,\ 1974 Introduction to the Theory of
Statistics (3rd ed.).\ McGraw-Hill.\ p.\ 229.
%https://archive.org/details/introductiontoth00mood_706/page/n241/mode/2up

\bibitem[Noferini and Townsend(2016)]{Noferini2016}
Noferini, V.,\ \& Townsend, A.\ 2016,
''Numerical instability of resultant methods for multidimensional rootfinding,''
SIAM Journal on Numerical Analysis, 54(2),\ 719--743.

\bibitem[Lam and Pitrou(2015)]{Numba}
Lam,\ S.\ K.,\ Pitrou, A.,\ \& Seibert,\ S.\ (2015).
''Numba: A llvm-based python jit compiler,''
In Proceedings of the Second Workshop on the LLVM Compiler Infrastructure in HPC
(pp.\ 1--6).

\bibitem[Oliphant(2006)]{Numpy}
Oliphant,\ T.\ E.\ (2006).\ A guide to NumPy (Vol.\ 1).\ Trelgol Publishing
USA .

\bibitem[Omohundro(1989)]{Balltree}
Omohundro,\ S.\ M.,\ (1989),
''Five balltree construction algorithms,''
International Computer Science Institute Technical Report

\bibitem[Pedregosa et al.(2011)]{sklearn}
Pedregosa,\ F.\ et al.,\ 2011.
''Scikit-learn: Machine learning in Python,''
Journal of machine learning research, 12(Oct), pp.2825--2830.

\bibitem[Rossum(1995)]{Python1995}
G. van Rossum, Python tutorial, Technical Report CS-R9526,
Centrum voor Wiskunde en Informatica (CWI), Amsterdam, May 1995.

\bibitem[Stone(1977)]{Stone1977}
Stone, C.J.\ 1977,
''Consistent Nonparametric Regression,''
Ann.\ Statist.,\ vol.\ 5 (1977), 595--645

\bibitem[Vacca et al.(2020)]{Vacca20}
Vacca, W. D. et al.\ 2020,
''The Data Reduction Pipeline for FIFI-LS, the MIR Integral Field Spectrograph
for SOFIA,''
ADASS 2019

\bibitem[scikit\-image(2014)]{skimage2014}
St\'{e}fan van der Walt, Johannes L. Sch\"{o}nberger, Juan Nunez-Iglesias,
Fran\c{c}ois Boulogne, Joshua D. Warner, Neil Yager, Emmanuelle Gouillart,
Tony Yu and the scikit-image contributors.\ 2014,
''scikit-image: Image processing in Python,''
PeerJ 2:e453

\end{thebibliography}

\end{document}
