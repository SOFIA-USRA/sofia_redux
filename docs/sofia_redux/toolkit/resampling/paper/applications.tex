\subsection{2-Dimensional Resampling of Regular Data}
\label{subsec:example-2d}

The following example uses a ($128 \times 128$ pixels) subset of the ``camera''
test data set from the Python scikit-image package (\citet{skimage2014}),
already displayed in figure~\ref{fig:adaptive-kernels} to show the effects of
the resampling algorithm applied to regularly spaced 2-D samples onto a finer
grid where $\Delta x = 0.2$ pixels.
The FWHM of the observing device was estimated to be 1 pixel (by eye), and the
error was estimated from the standard deviation of the blank sky region on the
upper-right corner of the image ($\approx 0.7\%$ of the maximum sample value).
The window radius is set to 12 pixels resulting in a median of 452 samples
per 3rd order polynomial fit.

Figure~\ref{fig:camera-fits} shows the sample image, and fits on a finer grid
using standard distance weighting with
$\alpha = 2 \left(\text{FWHM} / 2\sqrt{2 \ln 2}\right)^2$ and the results of
the scaled and shaped adaptive distance weighting algorithm.
In addition, different edge and order rejection algorithms were applied, the
effects of which can be seen on the borders of each fitted image.
The standard distance weighting algorithm rejected fits using the bounded
order requirement, while the adaptive fits allowed for extrapolation in cases
where enough unique samples are present.
However, the ``scaled'' solution applied an edge rejection threshold of
$\beta_{\text{edge}} = 1$.
In this specific case, edge and order rejection options are applied for the sake
of example only.
Good fits are possible due to regularly spaced sampling, well-behaved data,
and that we are not attempting any fits outside of the bounds of the sample
region.

\begin{figure}[H]
  \begin{center}
  \includegraphics[width=0.9\linewidth]{images/camera_fits.png}
  \caption{Resampling an image (upper-left) onto a finer grid using the
           standard distance weighting algorithm (upper-left), the ``scaled''
           adaptive weighting algorithm (lower-left), and the ``shaped''
           adaptive weighting algorithm (lower-right).  The red and green
           squares on the upper-left image mark regions shown in
           figure~\ref{fig:camera-samples}.}
  \label{fig:camera-fits}
  \end{center}
\end{figure}

At first glance, the standard and scaled solutions appear fairly similar to the
original image, while a strong reduction in the pixelation effects can be
seen in the shaped image.
The similarity of the standard and scaled solutions is cosmetic only,
and show notable differences when viewed on a smaller local scale.
Figure~\ref{fig:camera-samples} show enhanced sections of the fit on a
smaller scale which are marked for reference on the upper left image of
figure~\ref{fig:camera-fits}.

\begin{figure}[H]
  \begin{center}
  \includegraphics[width=0.8\linewidth]{images/camera_samples.png}
  \caption{Subsections of the sample image and fits marked by the regions in
           figure~\ref{fig:camera-fits}.  The upper row (``Feature''),
           contains an area with strong features (green box in
           figure~\ref{fig:camera-fits}), while the lower row (``Sky'') shows
           a relatively smooth portion of the sample data (red box in
           figure~\ref{fig:camera-fits}).  The columns from left to right
           display the sample image, the standard distance weighted fit, the
           scaled adaptive fit, and the shaped adaptive fit.}
  \label{fig:camera-samples}
  \end{center}
\end{figure}

\begin{figure}[H]
  \begin{center}
  \includegraphics[width=0.8\linewidth]{images/camera_chi2.png}
  \caption{Histograms of $\log_{10}(\chi_r^2)$ for each of the distance
           weighting algorithms.  The normalized count density is given as
           $\rho(N) = N_i / \left(\Delta P \sum_{i}{N_i}\right)$ where
           $\Delta P$ is the spacing between bins.}
  \label{fig:camera-chi2}
  \end{center}
\end{figure}

Finally, figure~\ref{fig:camera-chi2} displays histograms of the $\chi_r^2$
distribution for all fits over the entire resampled images (409,600 pixels).
On a ${\log}_{10}$ scale where $P = {\log}_{10}(\chi_r^2)$, the distribution can
be approximated by a normal distribution
$\mathcal{N}(\text{mean}(P), \text{variance}(P))$.
For our example fits $\mathcal{N}_{\text{standard}} = (-3.126, 0.905)$,
$\mathcal{N}_{\text{scaled}} = (0.198, 0.371)$, and
$\mathcal{N}_{\text{shaped}} = (0.092, 0.442)$.
In other words, the adaptive weighting algorithms produced fits where $\chi_r^2$
is more closely centered around 1 with smaller variance.

\subsection{3-Dimensional Resampling of Irregular Data}
\label{subsec:example-fifi-ls}

The following examples apply the new resampling algorithm to real data taken
with the FIFI-LS instrument on board SOFIA (see
section~\ref{subsec:motivation}).

\subsubsection{M82: A comparison of weighting schemes on a high SNR source}

Observations of M82 were taken over multiple SOFIA flights using the long
wavelength spectrometer.
Data are irregularly spaced in cross-elevation, elevation, and wavelength
($x, y, \lambda$) centered on $\lambda=157.694\mu m$, and J2000 coordinates (09h55m55.38s
+69d40m53.4s) with $x$ and $y$ in units of arcseconds.
The average spatial FWHM of the observations is
$\text{fwhm}_x = \text{fwhm}_y = 15.8\arcsec$, and the average spectral FWHM is
$\text{fwhm}_{\lambda} = 0.06689507\mu m$.
Histograms of the sample measurement values and errors are shown in
figure~\ref{fig:fifi-M82-sample-errors}.

The sample distribution is shown in
figure~\ref{fig:fifi-M82-sample-distribution}.
Note that the spatial (x, y) distribution consists of a sparsely sampled region
near the center bounded by two more densely sampled areas above and below, as
shown on the left image.
It can be seen from the right-most image that the wavelength sampling interval
For a single cluster of spatial coordinates shown on the left image as a single
blue dot, the median $\lambda$ sampling interval is $\Delta \lambda = 0.003482 \mu m \approx 19
\text{ samples}/\text{fwhm}_{\lambda}$.

\begin{figure}[H]
  \begin{center}
  \includegraphics[width=\linewidth]{images/fifi_M82_sample_errors.png}
  \caption{Histograms of the sample measurement values (left) and
  $1\sigma$ measurement errors (right).}
  \label{fig:fifi-M82-sample-errors}
  \end{center}
\end{figure}

\begin{figure}[H]
  \begin{center}
  \includegraphics[width=\linewidth]
      {images/fifi_M82_sample_distribution.png}
  \caption{Left: Spatial sample distribution in x (cross-elevation) and y
           (elevation) coordinates.  Right: Sample distribution displayed in
           x, y, and $\lambda$ (wavelength).}
  \label{fig:fifi-M82-sample-distribution}
  \end{center}
\end{figure}

Resampling was conducted to determine solutions for the standard, scaled, and
shaped distance weighting algorithms on a regular spaced ($x, y$)
($30 \times 72$) grid ($M=2,160$) with  spacing $\Delta x = \Delta y = 3\arcsec$
at constant $\lambda=157.83418\mu m$.
The window region was set to $\omega_{x,y}=\text{fwhm}_{x,y} \times 3$, and
$\omega_{\lambda} =\text{fwhm}_{\lambda} / 2$ resulting in a median of
4,200 samples per second order polynomial fit in a sample space containing
$N=147,200$ samples.
Fits were rejected according to the bounded order rejection algorithm, along
with an edge rejection threshold of $\beta_{\text{edge}} = 0.7$ in the spatial
dimensions, and 0.5 in $\lambda$.
For the standard distance weighting reduction, the smoothing parameter was
set to $\alpha_{x,y} = \text{fwhm}_{x,y}$, and
$\alpha_{\lambda} = \text{fwhm}_{\lambda} / 4$.
Adaptive weighting (scaled and shaped) was only applied to the spatial
dimensions, and fixed at $\alpha_{\lambda}$ in the wavelength dimension.

Figure~\ref{fig:fifi-M82-image} displays all three fits on the same scale,
whereas figure~\ref{fig:fifi-M82-log-image} displays each fit normalized to
within $10^{-3} \to 1$ on a log scale in order to better represent the
dynamic range.

Clear discontinuities can be seen from a visual inspection of the standard
distance weighting reduction in figure~\ref{fig:fifi-M82-image} in areas
where spatial sampling density appears to transition from one value to
another as shown in the left image of
figure~\ref{fig:fifi-M82-sample-distribution}.
However, both adaptive weighting algorithms appear smooth and continuous
in these density transition regions.

Reduction times on a MacBook Pro with a 2.5 GHz Intel Core i7 processor using
16 GB of memory is 0.6 seconds using the standard weighting algorithm, and
22 and 20 seconds for the ``shaped'' and ``scaled'' weighting algorithms,
respectively.

This factor in increased reduction time represents the fact that kernels are
calculated for each sample (in the neighborhoods of all resampling points)
rather than at each resampling point, and that $M \ll N$.
Generally, when $M \approx N$, the reduction time for an adaptive weighting
algorithm is roughly 2 to 3 times that of the standard weighting algorithm.

\begin{figure}[H]
  \begin{center}
  \includegraphics[width=0.8\linewidth]{images/fifi_M82_image.png}
  \caption{Images generated for $\lambda=157.83418\mu m$ with each available
           weighting algorithm.  Black dotted lines indicate regions centered
           on constant $x=4.06\arcsec$ and $y=1.43\arcsec$ of
           width $\text{fwhm}_{x,y}$.  The samples plotted in
           figure~\ref{fig:fifi-M82-cross-sections} reside in these regions,
           with the cross marking a line of constant $\lambda=157.83418\mu m$ through
           the image.}
  \label{fig:fifi-M82-image}
  \end{center}
\end{figure}

\begin{figure}[H]
  \begin{center}
  \includegraphics[width=0.8\linewidth]{images/fifi_M82_log_image.png}
  \caption{Images from figure~\ref{fig:fifi-M82-image} displayed on a log
           scale.}
  \label{fig:fifi-M82-log-image}
  \end{center}
\end{figure}

\newpage

Figure~\ref{fig:fifi-M82-cross-sections} displays fits in which the coordinates
for two dimensions are held fixed, and resampling occurs along the remaining
($x$, $y$, or $\lambda$) dimension as marked in the left image of
figure~\ref{fig:fifi-M82-image} (over the peak).
The scale of the Gaussian standard deviation of the weighting function is
plotted for each dimension.
In those dimensions in which adaptive weighting is enabled ($x, y$), the
minimum and maximum standard deviations of the ``scaled'' weighting kernels are
also drawn.
For reference, all samples within $\pm \text{fwhm} / 2$ are plotted with color
representing the Mahalanobis distance of a sample from the resampling point
in the fixed dimensions.
Color indicates the sample coordinate deviation of coordinates in the two
fixed dimensions with respect to the resampling point coordinate.
i.e., for the top plot in figure~\ref{fig:fifi-M82-cross-sections} along $x$,
where $y=1.43\arcsec$ and $\lambda=157.83418 \mu m$

\begin{equation}
    \Delta \sigma(y, \lambda) =
        \sqrt{\left( \frac{y - 1.43}{\sigma_y} \right)^2 +
              \left( \frac{\lambda - 157.83418}{\sigma_{\lambda}} \right)^2
        }
    \nonumber \label{eq:equation68}
\end{equation}

\begin{figure}[H]
  \begin{center}
  \includegraphics[width=0.75\linewidth]{images/fifi_M82_cross_sections.png}
  \caption{Cross sections of the fit along each dimension for each weighting
           scheme.  Top: A fit along the $x$ dimension, with $y$ and $\lambda$
           held constant.  Center: A fit along the $y$ dimension, with $x$ and
           $\lambda$ held constant.  Bottom: A fit along the $\lambda$
           dimension, with $x$ and $y$ held constant.  The intersection of all
           fixed dimensions is marked by a cross in the left image of
           figure~\ref{fig:fifi-M82-image}, and samples within the marked
           regions are plotted for reference.  Samples are color coded according
           to deviation of each sample to the resampling point in dimensions
           that are held constant.}
  \label{fig:fifi-M82-cross-sections}
  \end{center}
\end{figure}

\newpage

Figure~\ref{fig:fifi-M82-rchi2} shows the resulting $\chi_r^2$ distribution
for the fit along each dimension.
In all cases, $\chi_r^2 > 1$ indicating a poor fit.
However, this is the result of attempting a fit on data samples that cannot be
modelled within the expected $1\sigma$ noise limits supplied with the data.
From figure~\ref{fig:fifi-M82-sample-errors}, the typical supplied sample
measurement errors are $\epsilon=10 Jy$.
A quick look at figure~\ref{fig:fifi-M82-cross-sections} shows that the range
of values at a single coordinate can vary by approximately $\pm 200 Jy$ between
samples that occupy close spatial/spectral coordinates.

\begin{figure}[H]
  \begin{center}
  \includegraphics[width=0.9\linewidth]{images/fifi_M82_rchi2.png}
  \caption{Histograms of the $\chi_r^2$ values for each cross-section in $x$,
  $y$, and $\lambda$ at 5,000 regularly spaced resampling points
  along each dimension into 100 bins.}
  \label{fig:fifi-M82-rchi2}
  \end{center}
\end{figure}

For reference, a fit generated using standard distance weighting with
$\alpha = [\alpha_x, \alpha_y, \alpha_{\lambda}] / 12$ is shown in
figure~\ref{fig:fifi-M82-rchi2-equal-1} resulting in a $\chi_r^2$ approximately
centered around 1.
The resulting fit is unusable with no discernible structure and must be placed
on a $\log$ scale for visualization.

\begin{figure}[H]
  \begin{center}
  \includegraphics[width=\linewidth]{images/fifi_M82_chi2_equal_1.png}
  \caption{The result of fitting M82 sample data to $\approx \chi_r^2=1$.
           Left: Histogram of $\log_{10}(\chi_r^2)$ into 50 bins.
           Right: The image slice (also shown in
           figure~\ref{fig:fifi-M82-image}) when fit approximating
           $\chi_r^2=1$.  Colors represent flux (Jy) on a $\log_{10}$ scale
           of the absolute value.}
  \label{fig:fifi-M82-rchi2-equal-1}
  \end{center}
\end{figure}

\clearpage
\newpage

\subsubsection{30 Doradus: A comparison of simultaneous vs. sequenced
               dimensional resampling}

The 30 Doradus data set is intended to highlight some differences between the
old and new FIFI-LS data reduction pipelines, and the advantages of fitting
multiple dimensions simultaneously.
The initial pipeline sequentially resampled the data in wavelength (1 dimension)
followed by spatial resampling in two dimensions onto a regular grid.

Our example data set contains multiple observations of 30 Doradus
totalling 481,600 (x, y, $\lambda$) FIFI-LS sample measurements which are
resampled onto a ($327 \times 356 \times 50$) grid.
Unlike the previous M82 example, the sampling density is relatively uniform
near the area of interest, but the SNR ratio is much lower.
The sample distribution is displayed in
figure~\ref{fig:fifi-30DOR-sample-distribution}, and histograms of sample
measurement values and errors are shown in
figure~\ref{fig:fifi-30DOR-sample-errors}.

The central wavelength of the observations was at $\lambda = 51.87619 \mu m$
with $\text{FWHM}_{\lambda} = 0.056389 \mu m$, and
$\text{FWHM}_{x,y} = 6.2\arcsec$.
For both resampling algorithms, the window radius in the spectral dimension
is set to $\omega_{\lambda} = \text{FWHM}_{\lambda} / 2$.
When resampling in 3-dimensions simultaneously, the spatial window radius
is set to $\omega_{x,y} = 3 \times \text{FWHM}_{x, y}$.
Spectral resampling drastically reduces the number of samples available for
spatial resampling.
To allow for a smooth fit, the standard practice using the old pipeline was to
increase the spatial window radius to
$\omega_{x,y} = 6 \times \text{FWHM}_{x, y}$.
In both cases, a second order polynomial fit is generated along the spectral and
spatial dimensions using the standard distance weighting algorithm with a
smoothing parameter of
$\alpha_{\lambda} = \text{FWHM}_{\lambda} / 4$ and
$\alpha_{x,y} = 2 \times \text{FWHM}_{x,y}$.

\begin{figure}[H]
  \begin{center}
  \includegraphics[width=\linewidth]{images/fifi_30DOR_sample_distribution.png}
  \caption{Left: The spatial distribution of samples in x (cross-elevation),
           and y (elevation).  Right: The full 3D sampling distribution in
           x, y, and $\lambda$ (wavelength).}
  \label{fig:fifi-30DOR-sample-distribution}
  \end{center}
\end{figure}

\begin{figure}[H]
  \begin{center}
  \includegraphics[width=0.75\linewidth]{images/fifi_30DOR_sample_errors.png}
  \caption{Left: Histogram of the sample measurement values.  Right:
           Histogram of the associated $1\sigma$ measurement errors.}
  \label{fig:fifi-30DOR-sample-errors}
  \end{center}
\end{figure}


Many properties of the reduction are hard to compare due to major differences in
the way data are handled (for example, number of samples per fit or $\chi_r^2$).
We do, however, present comparisons of both a spatial and spectral slice that
intersect at the point of greatest emission.

Figure~\ref{fig:fifi-30DOR-image-comparison} displays a spatial slice at
$\lambda = 51.94550 \mu m$, and a spectral profile is shown in
figure~\ref{fig:fifi-30DOR-line-comparison} at (RA, Dec) =
(5h38m45.9046s, -69d05m06.2397s) which is also marked as a black cross on the
left image of figure~\ref{fig:fifi-30DOR-image-comparison}.

\begin{figure}[H]
  \begin{center}
  \includegraphics[width=\linewidth]{images/fifi_30DOR_image_comparison.png}
  \caption{Spatial slices at $\lambda=51.94550\mu m$.  Left: The result of
           first resampling in wavelength followed by the 2-dimensional spatial
           interpolation.
           Right: resampling in all dimensions simultaneously.
           The spatial position of the spectral profile shown in
           figure~\ref{fig:fifi-30DOR-line-comparison} is marked as a black
           cross on the left image.  The red circle encloses a prominent
           artifact.}
  \label{fig:fifi-30DOR-image-comparison}
  \end{center}
\end{figure}

\begin{figure}[H]
  \begin{center}
  \includegraphics[width=\linewidth]{images/fifi_30DOR_line_comparison.png}
  \caption{The spectral profile at RA = 5h38m45.9046s, Dec = -69d05m06.2397s.
           The red line shows the results of sequential resampling, while the
           blue line shows the results of directly resampling onto a
           3-dimensional grid.}
  \label{fig:fifi-30DOR-line-comparison}
  \end{center}
\end{figure}

It can easily be seen that resampling in all dimensions simultaneously using
the new FIFI-LS pipeline has produced a smoother image with fewer artifacts
than the older version, although peak fluxes are relatively equal.
Additionally, the spectral profile produced by the 3-dimensional fit is smooth
and continuous, whereas discontinuities are clearly visible when resampling
occurs sequentially over spectral and spatial dimensions.
In fact, the strong artifact shown by the red circle in
figure~\ref{fig:fifi-30DOR-image-comparison} is not consistent with slices
at neighboring wavelengths, and is colinear with many other artifacts that
are on the boundary of a change in sampling density as shown in
figure~\ref{fig:fifi-30DOR-sample-distribution}, indicating that it is not a
valid structure.
